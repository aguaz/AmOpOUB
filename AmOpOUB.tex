\documentclass{tufte-handout}

% --- Packages --- %
\usepackage[english]{babel}
\usepackage[utf8]{inputenc}
\usepackage{hyperref}
\usepackage{amsmath}
\usepackage{mathtools}
\usepackage{graphicx}
% prettier tables
\usepackage{booktabs}
% nice non-stacked fractions and better spacing
\usepackage{units} 
% allow subfile structure
\usepackage{subfiles}
% Small sections of multiple columns
\usepackage{multicol}
\usepackage{bbm}
\usepackage{enumitem}
% Manage large sidenotes
\usepackage{marginfix}
% Handle more floats
\usepackage{morefloats} 



% New Commands
\newcommand{\E}{\mathbb{E}} % Expected value
\renewcommand{\Pr}{\mathbb{P}} % Probability operator
\newcommand{\R}{\mathbb{R}} % Real numbers
\newcommand{\InfGen}{\mathbb{L}} % Infinitesimal generator
\newcommand{\Var}{{\mathbb{V}\rm ar}}
\newcommand{\Cov}{{\mathbb{C}\rm ov}}
\newcommand{\dif}{\mathrm{d}}
\newcommand{\red}[1]{\textcolor{red}{#1}}

\DeclareMathOperator{\csch}{csch}

% New environments
\newtheorem{pr}{Proposition}
\newtheorem{thm}{Theorem}
\newtheorem{alg}{Algorithm}
\newtheorem{df}{Definition}
\newtheorem{lm}{Lemma}
\newtheorem{cor}{Corollary}
\newtheorem{rmk}{Remark}
\newenvironment{pf}[1][Proof]{\textbf{#1.} }{\ \rule{0.5em}{0.5em}}

% Title and Author
\title{Optimal exercising time of American put options for an Orstein-Uhlenbeck bridge}
\author{Abel Guada}
% \date{28 January 2018}  % if the \date{} command is left out, the current date will be used

% If staments
\def\uniqtheo{1}



%%%%%%%%%%%%%%%%%%%%%
%%% %%% BEGIN %%% %%%
%%%%%%%%%%%%%%%%%%%%%

\begin{document}
	
	\maketitle% this prints the handout title, author, and date
	
	\begin{abstract}
		\noindent ABSTRACT 
	\end{abstract}
	
	
	\section{Introducction}
	
	Let $X = \{X_s\}_{s \geq 0}^{T}$ be an Orstein-Uhlenbeck Bridge\sidenote{Definition: let $Y = \{Y_s\}_{s\geq 0}^T$ be a Orstein-Uhlenbeck process. Then, $X = \{X_s\}_{s \geq 0}^{T}$ is said to be an OUB from $X_0 = Y_0$ to $X_T = z$ if $$\mathrm{Law}(X, \Pr) = \mathrm{Law}(Y, \Pr_{T, z}),$$ where $\Pr_{T, z} := \Pr(\cdot | Y_T = z)$.} (OUB) with final value $X_T = z$. It is well known (see, e.g., \citet{Barczy2013Sample}) that $X$ follows the dynamics\sidenote{We are setting to zero the parameter coming from the stationary mean of the underlying Orstein-Uhlenbeck process. The OUB is pulled towards this parameter if $\alpha < 0$ and pushed away if $\alpha > 0$.}
	\begin{align}\label{eq:OUB_SDE}
	\dif X_s = \mu(s, X_s)\mathrm{d}s + \sigma\mathrm{d}W_s,\ \ 0 \leq s \leq T.
	\end{align}
	living in the filtered space $(\Omega, \mathcal{F}, \Pr, \{\mathcal{F}_s\}_{s \geq 0}^T)$, where $\{\mathcal{F}_s\}_{s \geq 0}^T$ is the natural filtration of the underlying standard Brownian motion $\{W_s\}_{s\geq 0}^{T}$, and with\sidenote{The OUB does not get affected by the sign of $\alpha$, but rather by its absolute value.} $\alpha > 0$, $\sigma > 0$, and
	\begin{align}\label{eq:OUB_drift}
	\mu(t, x) = \alpha \frac{z - \cosh(\alpha(T - t)) x}{\sinh(\alpha(T - t))}.
	\end{align}
	
	Consider the finite horizon, discounted Optimal Stopping Problem (OSP)
	\begin{align}\label{eq:OSP}
	V(t, x) = \sup_{0\leq \tau \leq T - t}\E_{t, x}\left[e^{-\lambda \tau}G(X_{t + \tau})\right],
	\end{align}
	where $V$ is called the value function, $G(x) = (A - x)^+$, for some $A\in\R$, is the gain function, $\lambda \geq 0$ is the discounting rate, the supreme above is taken under all random times\sidenote{In what remains, we will call $\tau$ a stopping time while keeping in mind that $t + \tau$ is the real stopping time.} $\tau$ such that $t + \tau$ is a stopping time in $\{\mathcal{F}_s\}_{s \geq 0}^T$, and $\E_{t, x}$ represent the expectattion under the probability measure $\Pr_{t, x}$ define as $\Pr_{t, x}(\cdot) = \Pr(\cdot | X_t = x)$.
	
	The particular form of $G$ allows to come up with the following inequality,
	\begin{align}\label{eq:G_inequality}
	-(y - x)^- \leq G(x) - G(y) \leq (y - x)^+, 
	\end{align}
	for all $x, y \in \R$, which will be recurrently used throughout the paper.
	
	Sometimes it is useful to keep track of the condition $X_t = x$ in a way that does not change the probability measure whenever $t$ or $x$ change. To do so, we denote the process $X^{t, x} = \{X_s^{t, x}\}_{s\geq 0}^{T - t}$ in the filtered space $(\Omega, \mathcal{F}, \Pr, \{\mathcal{F}_s\}_{s\geq 0}^{T - t})$ such that
	\begin{align}\label{eq:lawX^{t,x}-X}
	\mathrm{Law}(\{X_s^{t, x}\}_{s\geq0}^{T - t}, \Pr) = \mathrm{Law}(\{X_{t + s}\}_{s\geq 0}^{T - t}, \Pr_{t, x}).	
	\end{align}
	Notice that $X_0^{t,x} = x$ $\Pr$-a.s. 
	
	Now we are able to define the processes $\partial_t X^{t, x} = \{\partial_t X_s^{t, x}\}_{s}^{T - t}$ and $\partial_x X^{t, x} = \{\partial_x X_s^{t, x}\}_{s}^{T - t}$ as the $\Pr$-a.s. limit
	\begin{align*}
	\partial_t X_s^{t, x} &:= \lim_{\varepsilon\rightarrow 0} \left(X_s^{t + \varepsilon, x} - X_s^{t, x}\right)\varepsilon^{-1}, \\
	\partial_x X_s^{t, x} &:= \lim_{\varepsilon\rightarrow 0} \left(X_s^{t, x + \varepsilon} - X_s^{t, x}\right)\varepsilon^{-1}.
	\end{align*}
	
	One can compute the strong solution of \eqref{eq:OUB_SDE} (see \citet{Barczy2013Sample}) and use the Markov property of $(t, X_t)$ alongside \eqref{eq:lawX^{t,x}-X} to be able to set
	\begin{align}\label{eq:OUB_rep_int}
	X_{s}^{t, x} = x\frac{\sinh(\alpha (T - t - s))}{\sinh(\alpha (T - t))} &+ z\frac{\sinh(\alpha s)}{\sinh(\alpha (T - t))}  \\ 
	&+ \sigma\int_{0}^s \frac{\sinh(\alpha (T - t - s))}{\sinh(\alpha (T - t - u))}\,\mathrm{d}W_u, \nonumber
	\end{align}
	and then\sidenote{\red{ToDo:} I'm not entirely sure about the value of $\partial_t X_s^{t, x}$ for $s = T - t$. I guess is zero as the $X_{T-t}^{t, x} = z$ for all $t$.},
	\begin{align}
	\partial_t X_s^{t, x} &= \begin{cases}
	\alpha\left(\frac{\sinh(\alpha s)(z\cosh(\alpha(T - t)) - x)}{\sinh^2(\alpha(T - t))} - \sigma M_s^t\right), & s < T - t \\
	0, & s = T - t 
	\end{cases} \label{eq:OUB_t_rep_int} \\
	\partial_x X_s^{t, x} &= \frac{\sinh(\alpha (T - t - s))}{\sinh(\alpha (T - t))}, \label{eq:OUB_x_rep_int}
	\end{align}
	where $M_s^t$ takes the form
	\begin{align}\label{eq:Ms}
	M_s^{t} = \int_{0}^{s}\frac{\sinh(\alpha (s - u))}{\sinh^2(\alpha(T - t - u))}\dif W_u.
	\end{align}
	Also, it follows from \eqref{eq:OUB_SDE} that the infinitesimal generator $\InfGen$ of the process $\{(t, X_t)\}_{t\geq 0}^T$ is given by
	\begin{align}\label{eq:OUB_InfGen}
	(\InfGen f)(t, x) = \partial_t f(t, x) + \mu(t, x)\partial_x f(t, x) + \frac{\sigma^2}{2} \partial_{xx}f(t, x),
	\end{align}
	where we use $\partial_t$ and $\partial_x$ to represent the partial derivative with respect to time and space respectively, and $\partial{xx}$ is a shorthand of $\partial_x\partial_x$.
	
	Denote by $D := \{V = G\}$ and $C := D^c =\{V > G\}$ the so-called stopping set and continuation set respectively. Since\sidenote{Should I proof this? or it is trivial enough to be just stated?} 
	\begin{align}\label{eq:Dominated_convergence_condition}
	\E_{t, x}\left[\sup_{0\leq s\leq T - t} e^{-\lambda s}G(X_{t + s})\right] < \E_{t, x}\left[\sup_{0\leq s\leq T - t} |X_{t + s}|\right] < \infty
	\end{align} 
	for all $(t, x) \in [0, T)\times\R$, we can guarantee\sidenote{Citation!} that, under $\Pr_{t, x}$, the first hitting time of $\{X_{t + s}\}_{s\geq0}^{T - t}$ into $D$, denoted by $\tau^* = \tau^*(t, x)$, is optimal in \eqref{eq:OSP}, meaning that
	\begin{align}\label{eq:OST}
	V(t, x) = \E_{t, x}\left[e^{-\lambda\tau^*}G(X_{t + \tau^*})\right].
	\end{align}
	Moreover\sidenote{I think we don't use this property.}, if there is another Optimal Stopping Time (OST) $\tau$, then $\tau^* \leq \tau\ \ \Pr_{t, x}$-a.s.
	
	Also, it is handy to keep in mind the following martingale and supermartingale properties of $V$,
	\begin{align}
	\E_{t, x}\left[V(t + \tau^*\wedge s, X_{t + \tau^*\wedge s})\right] &=  V(t, x), \label{eq:V_martingale} \\
	\E_{t, x}\left[V(t + s, X_{t + s})\right] &\leq V(t, x), \label{eq:V_supermartingale}
	\end{align}	
	for all $0 \leq s < T - t.$
	
	Due to to explosive nature of the drift \eqref{eq:OUB_drift} near $T$, sometimes it is easier to derive properties about the the $\varepsilon$-truncated OSP, for $\varepsilon > 0$,
	\begin{align}\label{eq:OSP_eps}
	V^{\varepsilon}(t, x) = \sup_{\tau \leq T_\varepsilon - t}\E_{t, x}\left[e^{-\lambda\tau}G(X_{t + \tau})\right],
	\end{align}
	with $T_\varepsilon := T - \varepsilon$ and $(t, x) \in [0, T_\varepsilon]\times\R$, and then use them to shed light on the OSP \eqref{eq:OSP} by letting $\varepsilon \downarrow 0$, than it is to directly approach \eqref{eq:OSP}. Define $D^\varepsilon$ and $C^\varepsilon$ as the stopping and continuation sets, respectively, associated to \eqref{eq:OSP_eps}. Notice that \eqref{eq:Dominated_convergence_condition} ensures that the first hitting time of the process into $D^\varepsilon$, denoted by $\tau_\varepsilon^*(t, x)$ under $\Pr_{t, x}$, is optimal in \eqref{eq:OSP_eps}. Also, the analogous version of \eqref{eq:OST}, \eqref{eq:V_martingale}, and \eqref{eq:V_supermartingale}, also are satisfied by the $\varepsilon$-truncated value function $V^\varepsilon$ and OST $\tau_\varepsilon^*$. 
	
	\section{Regularities of the boundary and the value function}
	
	\begin{pr}\label{pr:boundary_existence}(Boundary existence and the shape of the stopping set)\\
		There exists a function $b:[0, T]\rightarrow\mathbb{R}$ such that $\infty < b(t) < A$ for all $t\in[0, T)$, $b(T) = \min\{A, z\}$, and $D = \{(t, x)\in[0, T]\times\mathbb{R} : x \leq b(t)\}$.
	\end{pr}
	
	\begin{pf}
		$(\mathcal{A}^+ \in C)$\sidenote{This fact remains true for all regular processes.}\ \ Let $(t, x)\in \mathcal{A}^+ := [0, T)\times[A, \infty)$, and consider the stopping time $\tau_\varepsilon = \inf\{s\in[0, T - t] :\ X_{t + s} \leq A - \varepsilon\}$ for some $\varepsilon > 0$ assuming that $\inf\{\emptyset\} = T - t$. Then, since $\tau_{\varepsilon} \leq T - t$ and $p := \mathbb{P}(\tau_\varepsilon < T - t) > 0$,
		\begin{align*}
		V(t, x) \geq \mathbb{E}_{t, x}\left[e^{-\lambda\tau_{\varepsilon}}G(X_{t + \tau_{\varepsilon}})\right] \geq e^{-\lambda(T - t)}p\varepsilon > 0 = G(x),
		\end{align*}
		this is, $(t, x)\in C$. Therefore $\mathcal{A}^+ \subset C$.	\\ \vspace{0.3cm}
		
		$(D = \{(t, x) : x \leq b(t)\})$\sidenote{Relies on $^xX_s^{T - t, z} \leq\ ^{x'}X_s^{T - t, z}$ for all $s\in [0, T - t]$ a.s. whenever $x \leq x'$, i.e. the process is decreasing with respect to the spacial initial condition.}\ \ Define $b(t) := \sup\{x : (t, x) \in D\}$. We have proved that $b(t) < A$ for all $t\in[0, T)$ and it is obvious that $b(T) = \min\{A, z\}$. On the other hand, due to $X_s^{t, x} \leq\ X_s^{t, x'}$ for all $s\in [0, T - t]$ $\Pr$-a.s. whenever $x \leq x'$, we can ensure that $D$ lies bellow the curve $b$. Moreover, since $D$ is closed, $(t, b(t)) \in D$ for all $t\in[0, T]$, which guarantees that $D$ has the claimed shape. \\  \vspace{0.3cm}
		
		$(\infty < b(t))$\ \ The from of the drift $\mu(t, x)$ of the OUB in \eqref{eq:OUB_drift} guarantees the existence of a function $u:[0, T] \rightarrow R$ such that $\mu(t, x) \geq 0$ for all $x \leq u(t)$ and $\mu(t, x) \leq 0$ for all $x \geq u(t)$. Define $m := \min_{t\in[0,T)}\{u(t)\}$. Notice that
		\begin{align*}
		X_t &\geq u(t) - \left|u(t) - X_t\right| \\
		& \geq u(t) - \left|u(t) - \sigma B_t\right| \\
		& \geq m - \left|m - \sigma B_t\right|,
		\end{align*}  
		where $B_t = X_0 + W_t$ and $\{W_t\}_{t\geq0}$ is the underlying standard Brownian motion in \eqref{eq:OUB_SDE}. The second inequality holds since the drift of the reflected (with respecto to $u$) OUB is always higher than the drift of the reflected (with respecto to $u$) Brownian motion, and therefore we can ensure that the first process is greater than the last one path-wise $\Pr$-a.s. (see \citet{ikeda1977}). The above inequalities guarantee that the value function \eqref{eq:OSP} is lower than the value function associated to a reflected (with respect to $m$) Brownian motion, and hence the respective OSBs hold the reverse inequality. On the other hand, it is easy to show that the OSB for the dsicounted OSP with the gain function $G$ and a reflected (with respect to $m$) Brownian motion is finite. Actually, one can obtain\sidenote{Done in Mathematica} explicitly the OSB, which is constant, by directly solving the associated free boundary problem. Therefore the OSB $b$ is also finite.
	\end{pf}
	
%	\begin{pr}(Regularity of $V$)\label{pr:V_regularity}\\
%		The value function $V$ satisfies:
%		\begin{enumerate}[label=(\roman{*}), ref=(\textit{\roman{*}})]
%			\item \label{pr:V_Lcontinuous} $V$ is Lipschitz continuous on $[0, T - \delta]\times\R$ for every $0 < \delta < T$. 
%			\item \label{pr:V_C^12onC} $V$ is $\mathcal{C}^{1,2}$ on $C$ and on $D$, and $\partial_tV + \mathbb{L}_{X}V = \lambda V$ on $C$. 
%			\item \label{pr:V_x-V_t} The following relations hold for all $(t, x) \in [0, T)\times\R$ such that $x\neq b(t)$,
%			\begin{align}
%			\partial_t V(t, x) &\leq \E\left[\left(-\partial_t X_{\tau^*}^{t, x}\right)^+\right], \label{eq:Vt_up} \\		
%			\partial_t V(t, x) &\geq - \E\left[\left(-\partial_t X_{\tau^*}^{t, x}\right)^-\right] \nonumber \\
%			&\ \ \ \ - \E\left[\mathbb{1}(\tau^* = T - t)\left(\alpha^2\frac{x + N^t - z}{\sinh(\alpha (T - t))}\right)^-\right], \label{eq:Vt_low}
%			\end{align}
%			and
%			\begin{align}\label{eq:V_x}
%			\partial_xV(t, x) = -\E\left[e^{-\lambda\tau^*}\partial_x X_{\tau^*}^{t, x}\right],
%			\end{align} 
%			where $\tau^* = \tau^*(t, x)$ and
%			\begin{align}\label{eq:N^t}
%			N^t = \sigma\int_{0}^{T - t} \frac{\sinh(\alpha (T - t))}{\sinh(\alpha (T - t - u))}\,\mathrm{d}W_u. 
%			\end{align} 			
%		\end{enumerate}
%	\end{pr}
%	
%	\begin{pf}
%		\ref{pr:V_Lcontinuous}\ \ \ Due to the martingale and super-martingale properties of $V$, \eqref{eq:V_martingale} and \eqref{eq:V_supermartingale} respectively, the fact that $V \geq G$ on $[0, T)\times\R$ and $V = G$ on $D$, and inequality \eqref{eq:G_inequality}, one has the following for $0 < \varepsilon \leq t \leq T - \delta$ and $\tau^* = \tau^*(t, x)$, 
%		\begin{align}
%		V(t, x) - V(T_\varepsilon, x) &\leq \E_{t, x}\left[V(t + \tau^*, X_{t + \tau^*})\right] - \E_{T_\varepsilon, x}\left[V(T_\varepsilon + \tau^*, X_{t + \tau^*})\right] \nonumber \\
%		&\leq \E_{t, x}\left[G(X_{t + \tau^*})\right] - \E_{T_\varepsilon, x}\left[G(X_{t + \tau^*})\right] \nonumber \\
%		&= \E\left[G\left(X_{\tau^*}^{t, x}\right) - G\left(X_{\tau^*}^{T_\varepsilon, x}\right)\right] \nonumber \\
%		&\leq \E\left[\left(X_{\tau^*}^{T_\varepsilon, x} - X_{\tau^*}^{t, x}\right)^+\right] \nonumber \\
%		&= \varepsilon\E\left[\left(-\partial_tX_{\tau^*}^{t_\varepsilon, x}\right)^+\right] \label{eq:Vt-Vt-eps<} \\
%		&\leq \varepsilon L_{1, x, \delta}, \nonumber
%		\end{align}
%		where the last equality holds for some $t_\varepsilon\in(T_\varepsilon, t)$ due to the mean value theorem, and
%		$$
%		L_{1, x, \delta} = \sup_{0\leq t \leq T - \delta}\E\left[\sup_{0\leq s \leq T - t}|\partial_tX_s^{t, x}|\right].
%		$$
%		
%		On the other hand, considering $0 < \varepsilon \leq T - t - \delta$, using similar arguments as the ones held in \eqref{eq:Vt-Vt-eps<}, and noticing that $\tau^*\wedge(T - T_\varepsilon)$ is admissible for $V(t + \varepsilon, x)$, we get
%		\begin{align}
%		V(t + \varepsilon, x) - V(t, x) &\geq \E_{t + \varepsilon, x}\left[V\left(t + \varepsilon + \tau^*\wedge(T - T_\varepsilon), X_{t + \tau^*\wedge(T - T_\varepsilon)}\right)\right] - \E_{t, x}\left[V\left(t + \tau^*, X_{t + \tau^*}\right)\right] \nonumber \\
%		&\geq \E_{t + \varepsilon, x}\left[G\left(X_{t + \tau^*\wedge(T - T_\varepsilon)}\right)\right] - \E_{t, x}\left[G\left(X_{t + \tau^*}\right)\right] \nonumber \\
%		&= \E\left[G\left(X_{\tau^*\wedge(T - T_\varepsilon)}^{t + \varepsilon, x}\right) - G\left(X_{\tau^*}^{t, x}\right)\right] \nonumber \\
%		&\geq \E\left[-\left(X_{\tau^*}^{t, x} - X_{\tau^*\wedge(T - t -\varepsilon)}^{t + \varepsilon, x}\right)^-\right] \nonumber \\
%		&= -\E\left[\mathbb{1}(\tau^*\leq T - T_\varepsilon)\left(X_{\tau^*}^{t, x} - X_{\tau^*}^{t + \varepsilon, x}\right)^-\right] - \E\left[\mathbb{1}(\tau^* > T - T_\varepsilon)\left(X_{\tau^*}^{t, x} - z\right)^-\right]. \label{eq:Vt+eps-Vt>}
%		\end{align}
%		Notice that
%		$$
%		\E\left[\mathbb{1}(\tau^*\leq T - T_\varepsilon)\left(X_{\tau^*}^{t, x} - X_{\tau^*}^{t + \varepsilon, x}\right)^-\right] \leq \varepsilon L_{1, x, \delta},
%		$$
%		and
%		\begin{align}
%		&\E\left[\mathbb{1}(\tau^* > T - T_\varepsilon)\left(X_{\tau^*}^{t, x} - z\right)^-\right] \nonumber \\
%		&\leq \E\left[\sup_{0 \leq h\leq \varepsilon}\left|X_{T - t - h}^{t, x} - z\right|\right] \nonumber \\
%		&= \E\left[\sup_{0 \leq h\leq \varepsilon}h\left|\alpha^2\frac{\cosh(h')\left(x + \sigma\int_{0}^{T - t -h} \frac{\sinh(\alpha (T - t))}{\sinh(\alpha (T - t - u))}\,\mathrm{d}W_u\right) - z\cosh(h'')}{\sinh(\alpha (T - t))}\right|\right] \nonumber \\
%		&\leq \varepsilon L_{2, x, \delta} \nonumber
%		\end{align}
%		where the equality results after considering \eqref{eq:OUB_rep_int} and applying the mean value theorem to get $h'$ and $h''$ such that they are both contained in $[-|\alpha|T, |\alpha|T]$. We then relied on the fact that $h \leq \varepsilon$ to come up with the bound in the last inequality, with
%		$$
%		L_{2, x, \delta} = \sup_{0\leq t \leq T - \delta}\E\left[\sup_{\substack{0 \leq h \leq T - t \\ 0 \leq h', h'' \leq |\alpha|T}}\left|\alpha^2\frac{\cosh(h')\left(x + \sigma\int_{0}^{T - t - h} \frac{\sinh(\alpha (T - t))}{\sinh(\alpha (T - t - u))}\,\mathrm{d}W_u\right) - z\cosh(h'')}{\sinh(\alpha (T - t))}\right|\right].
%		$$
%		
%		Now consider $\tau_\varepsilon = \tau^*(T_\varepsilon, x)$ for $0 \leq \varepsilon \leq t \leq T - \delta$, and notice that $\tau_\varepsilon\wedge(T - t)$ is admissible for $V(t, x)$. Then, argue as in \eqref{eq:Vt+eps-Vt>} to get
%		\begin{align*}
%		V(t, x) - V(T_\varepsilon, x)
%		&\geq -\E\left[\mathbb{1}(\tau_\varepsilon\leq T - t)\left(X_{\tau_\varepsilon}^{T_\varepsilon, x} - X_{\tau_\varepsilon}^{t, x}\right)^-\right] - \E\left[\mathbb{1}(\tau_\varepsilon > T - t)\left(X_{\tau_\varepsilon}^{T_\varepsilon, x} - z\right)^-\right],
%		\end{align*}
%		with
%		$$
%		\E\left[\mathbb{1}(\tau_\varepsilon\leq T - t)\left(X_{\tau_\varepsilon}^{T_\varepsilon, x} - X_{\tau_\varepsilon}^{t, x}\right)^-\right] \leq \varepsilon L_{1, x, \delta},
%		$$ 
%		and
%		\begin{align*}
%		&\E\left[\mathbb{1}(\tau_\varepsilon > T - T_\varepsilon)\left(X_{\tau_\varepsilon}^{t, x} - z\right)^-\right] \\
%		&\leq \E\left[\sup_{0 \leq r\leq \varepsilon}\left|X_{T - t + r}^{t, x} - z\right|\right] \\
%		&= (\varepsilon - r)\E\left[\sup_{0 \leq r\leq \varepsilon}\left|\alpha^2\frac{\cosh(r')\left(x + \sigma\int_{0}^{T - t + r} \frac{\sinh(\alpha (T - t + \varepsilon))}{\sinh(\alpha (T - t + \varepsilon - u))}\,\mathrm{d}W_u\right) - z\cosh(r'')}{\sinh(\alpha (T - t + \varepsilon))}\right|\right] \\
%		&\leq  \varepsilon L_{3, x, \delta},
%		\end{align*}
%		with
%		$$
%		L_{3, x, \delta} = \sup_{0\leq t\leq T - \delta} \E\left[\sup_{\substack{0 \leq r, \varepsilon\leq t \\ 0 \leq r', r''\leq T}}\left|\alpha^2\frac{\cosh(r')\left(x + \sigma\int_{0}^{T - t + r} \frac{\sinh(\alpha (T - t + \varepsilon))}{\sinh(\alpha (T - t + \varepsilon - u))}\,\mathrm{d}W_u\right) - z\cosh(r'')}{\sinh(\alpha (T - t))}\right|\right].
%		$$
%		
%		Additionally, for $0 \leq \varepsilon \leq T - t $ and $\tau^\varepsilon = \tau^*(t + \varepsilon, x)$ one gets the following by proceeding as in \eqref{eq:Vt-Vt-eps<}
%		\begin{align}
%		V(t + \varepsilon, x) - V(t, x) 
%		&\leq \varepsilon\E\left[\left(-\partial_tX_{\tau^\varepsilon}^{t^\varepsilon, x}\right)^+\right] \nonumber \\
%		&\leq  \varepsilon L_{1, x, \delta},
%		\end{align}
%		for some $t^{\varepsilon}\in(t, t + \varepsilon)$.
%		
%		So far, as \sidenote{Trivial enough?} $L_{i, x, \delta} < \infty$ for $i =1, 2, 3$, we have proved that, for any $\varepsilon$ satisfying $|\varepsilon| \leq t \leq T - \delta$, there exists a constant $L_{x,\delta} < \infty$ such that
%		$$
%		|V(t + \varepsilon, x) - V(t, x)| \leq \varepsilon L_{x, \delta},
%		$$
%		which means that $t\mapsto V(t, x)$ is Lipschitz continuous in the interval $[0, T - \delta]$. We will now prove that $V$ is also Lipschitz continuous with respect to $x$ for all $t\in [0, T)$, which will completes the proof of \ref{pr:V_Lcontinuous}.
%		
%		Since $G(x)$ decreases and $X_s^{t, x}$ increases as $x$ grows for all $s\in[0, T - t)$, then $V(t, x)$ decreases on $x$ for all $t\in[0, T)$. Fix $(t, x)\in[0, T]\times\R$ and $\varepsilon > 0$. Consider $\tau^* = \tau^*(t, x)$, and combine \eqref{eq:OSP}, \eqref{eq:OST}, \eqref{eq:G_inequality}, and \eqref{eq:OUB_rep_int}, to get 
%		\begin{align}
%		0 \geq V(t, x + \varepsilon) - V(t, x) &\geq \E_{t, x + \varepsilon}\left[e^{-\lambda\tau^*}G(X_{t + \tau^*})\right] - \E_{t, x}\left[e^{-\lambda\tau^*}G(X_{t + \tau^*})\right] \nonumber \\
%		&= \E\left[e^{-\lambda\tau^*}\left(G(X_{\tau^*}^{t, x + \varepsilon}) - G(X_{\tau^*}^{t, x})\right)\right] \nonumber\\
%		&\geq -\E\left[e^{-\lambda\tau^*}\left(X_{\tau^*}^{t, x + \varepsilon} - X_{\tau^*}^{t, x}\right)^+\right] \nonumber\\
%		&= -\varepsilon\E\left[e^{-\lambda\tau^*}\frac{\sinh(\alpha (T - t - \tau^*))}{\sinh(\alpha (T - t))}\right] \label{eq:Vx+eps-Vx}  \\
%		&\geq -\varepsilon \nonumber
%		\end{align}
%		Similar arguments with $\sigma_\varepsilon = \tau^*(t, x - \varepsilon)$ yield
%		\begin{align*}
%		0 \geq V(t, x) - V(t, x - \varepsilon) &\geq -\varepsilon\E\left[e^{-\lambda\sigma_\varepsilon}\frac{\sinh(\alpha (T - t - \sigma_\varepsilon))}{\sinh(\alpha (T - t))}\right] \geq -\varepsilon
%		\end{align*}
%		Then, $x\mapsto V(t, x)$ is Lipschitz continuous for all $t\in[0, T)$, which alongside the Lipschitz continuity of $t\mapsto V(t, x)$	in $[0, T - \delta]$ for all $0 < \delta < T$, allows us to conclude that $V$ is Lipschitz continuous on $[0, T - \delta]\times\R$, this is, for any $0 < \delta < T$ there exists $L_\delta$ such that
%		\begin{align*}
%		|V(t_1, x_1) - V(t_2, x_2)| \leq L_\delta(|t_1 - t_2| + |x_1 - x_2|),
%		\end{align*} 
%		for all $(t_1, x_1), (t_2, x_2) \in [0, T - \delta]\times\R$.\\  \vspace{0.3cm}
%		
%		
%		\ref{pr:V_C^12onC}\sidenote{This part is valid whenever $V$ is continuous in $C$ and both $\mu$ and $\sigma$ are locally $\alpha$-Hölder continuous in $[0, T)\times\R$}\ \ The fact that $\partial_tV + \mathbb{L}_{X}V = \lambda V$  on $C$ comes right after the strong Markov property of $(t, X_t)$ (See \citet[Section 7.1]{goran-optimal} for more details).
%		
%		Also, since $V$ is continuous in $C$ (see \ref{pr:V_Lcontinuous}), $\mu$ in $\eqref{eq:OUB_drift}$ is locally $\alpha$-Hölder continuous in $[0, T)\times\R$, and $\sigma$ is constant, then, one can borrow standard result from parabolic PDE's theory (see \citet[Section 3, Theorem 9]{friedman1964partial}) to guarantee that, for an open rectangle $R\subset C$, the first initial-boundary value problem
%		\begin{align}
%		\partial_t f + \mathbb{L}_{X} f - \lambda f &= 0 && \text{in } R, \label{eq:PDE1} \\
%		f &= V && \text{on } \partial R, \label{eq:PDE2}
%		\end{align}
%		has an unique solution $f$ such that $f\in \mathcal{C}^{1, 2}(R)$. Therefore, we can use Itô's formula on $f(X_{t + s})$ at $s = \tau_{R^c}$, this is, the first time $X_{t + s}$ exits $R$, and then take $\Pr_{t, x}$-expectation with $x \in R$, which guarantees the vanishing of the martingale term and yields, alongside both (\ref{eq:PDE1}) and (\ref{eq:PDE2}), the equality $\E_{t, x}[V(X_{t + \tau_{R^c}})] = f(t, x)$. Finally, notice that, due to the strong Markov property, $\E_{t, x}[V(X_{t + \tau_{R^c}})] = V(t, x)$.
%		\vspace{0.3cm}
%		
%		\ref{pr:V_x-V_t}\ \ As we have just proved in \ref{pr:V_C^12onC}, the partial derivatives of $V$ exist on $C$, and trivially they also exist on $D^{\circ}$ since $V = G$ on $D$. Therefore, for any $(t, x)\in [0, T)\times \R$ with $x\neq b(t)$, we can take $\varepsilon\rightarrow 0$ in \eqref{eq:Vt-Vt-eps<}, \eqref{eq:Vt+eps-Vt>}, and \eqref{eq:Vx+eps-Vx}, to derive equations \eqref{eq:Vt_up}, \eqref{eq:Vt_low}, and \eqref{eq:V_x} respectively.		
%	\end{pf}
		
	\begin{pr}($V^{\varepsilon}$ converges uniformly to $V$ in compacts) \label{pr:V^eps->V}\\
		For all compact $K \in [0, T)\times\R$ there exist a constant $L_K > 0$ such that $|V(t, x) - V^{\varepsilon}(t, x)| \leq \varepsilon L_K$ whenever $(t, x) \in K$.
	\end{pr}
	
	\begin{pf}
		It is both intuitive and pretty easy to check that $V(t, x) \geq V^\varepsilon(t, x)$ for $(t, x) \in [0, T_\varepsilon]\times\R$. Let $\tau^* = \tau^*(t, x)$ be the OST for \eqref{eq:OSP} and notice that $\tau^*\wedge(T_\varepsilon - t)$ it is an admissible stopping time for \eqref{eq:OSP_eps}. Therefore, due to the martingale and supermartingale properties of the value functions $V$ and $V^{\varepsilon}$, we get
		\begin{align}
		0 &\leq V(t, x) - V^\varepsilon(t, x) \nonumber \\
		&\leq \E_{t, x}\left[V(t + \tau^*, X_{t + \tau^*}) - V^\varepsilon(t + \tau^*\wedge(T_\varepsilon - t))\right] \nonumber \\
		&= \E_{t, x}\left[\mathbb{1}(\tau^* < T_\varepsilon - t)\left(V(t + \tau^*, X_{t + \tau^*}) - V^\varepsilon(t + \tau^*, X_{t + \tau^*})\right)\right] \nonumber \\
		&\ \ \ \ + \E_{t, x}\left[\mathbb{1}(\tau^* \geq T_\varepsilon - t)\left(V(t + \tau^*, X_{t + \tau^*}) - V^\varepsilon(T_\varepsilon, X_{T_\varepsilon})\right)\right] \nonumber \\
		&\leq \E_{t, x}\left[\mathbb{1}(\tau^* \geq T_\varepsilon - t)\left(V(t + \tau^*, X_{t + \tau^*}) - V^\varepsilon(T_\varepsilon, X_{T_\varepsilon})\right)\right] \nonumber \\ 
		&\leq \E_{t, x}\left[\mathbb{1}(\tau^* \geq T_\varepsilon - t)\left(V(t + \tau^*, X_{t + \tau^*}) - G(X_{T_\varepsilon})\right)\right] \nonumber \\
		&= \E_{t, x}\left[\mathbb{1}(\tau^* \geq T_\varepsilon - t)\left(\E_{t, x}\left[V(t + \tau^*, X_{t + \tau^*}) | \mathcal{F}_{T_\varepsilon - t}\right] - G(X_{T_\varepsilon})\right)\right] \nonumber \\
		&= \E_{t, x}\left[\mathbb{1}(\tau^* \geq T_\varepsilon - t)\left(V(T_\varepsilon, X_{T_\varepsilon}) - G(X_{T_\varepsilon})\right)\right]. \label{eq:V^eps->V_}
		\end{align}
		The definition of $V$ in \eqref{eq:OSP} alongside the Itô-Tanaka formula guarantee \sidenote{Is it $\frac{\varepsilon}{2}$ or $\sigma\frac{\varepsilon}{2}$?}
		\begin{align}
			V(T_\varepsilon, X_{T_\varepsilon}) &= \sup_{\tau \leq \epsilon}\E_{T_\varepsilon, X_{T_\varepsilon}}\left[e^{-\lambda\tau}G(X_{T_\varepsilon + \tau})\right] \nonumber \\
			&= G(X_{T_\varepsilon}) - \sup_{\tau \leq \epsilon}\E_{T_\varepsilon, X_{T_\varepsilon}}\left[\int_{0}^{\tau}\mathbb{1}(X_{T_\varepsilon + u}< A)e^{-\lambda u}\left(A - X_{T_\varepsilon + u} + \mu(T_\varepsilon + u, X_{T_\varepsilon + u})\right)\,\dif u\right. \nonumber \\
			&\hspace{4cm}\left. \frac{1}{2}\int_{0}^{\tau}\mathbb{1}(X_{T_\varepsilon + u} = A)\,\dif l_u^A(X_{T_\varepsilon + \cdot}) \right] \nonumber \\
			&\leq G(X_{T_\varepsilon}) + \E_{T_\varepsilon, X_{T_\varepsilon}}\left[\int_{0}^{\varepsilon}\left|A - X_{T_\varepsilon + u} + \mu(T_\varepsilon + u, X_{T_\varepsilon + u})\right|\,\dif u\right] + \frac{\varepsilon}{2} \nonumber \\
			&\leq G(X_{T_\varepsilon}) + \varepsilon \sup_{0 \leq t \leq T}\E_{t, X_{t}}\left[\sup_{t \leq s\leq T}\left|A - X_{s} + \mu(s, X_{s})\right|\right] + \frac{\varepsilon}{2}, \label{eq:V^eps->V_1}
		\end{align}		
		where $l_u^A(X)$ is the local time at $S$ of the process $X$
		$$
		l_u^A(X) = \lim_{h\downarrow0}\frac{1}{h}\int_{0}^{u}\mathbb{1}(A - h \leq X_s \leq A + h)\,\dif\langle X, X\rangle_s.
		$$
		Plugging \eqref{eq:V^eps->V_1} into \eqref{eq:V^eps->V_} we get
		\begin{align*}
			0 &\leq V(t, x) - V^\varepsilon(t, x) \nonumber \\
			&\leq \E_{t, x}\left[\mathbb{1}(\tau^* \geq T_\varepsilon - t)\left(\varepsilon \E_{t, X_{t}}\left[\sup_{t \leq s\leq T}\left|A - X_{s} + \mu(s, X_{s})\right|\right] + \frac{\varepsilon}{2}\right)\right] \\
			&\leq \varepsilon L_K,           
		\end{align*}
		with
		$$
		L_K = \sup_{(t, x) \in K}\E_{t, x}\left[\E_{t, X_{t}}\left[\sup_{t \leq s\leq T}\left|A - X_{s} + \mu(s, X_{s})\right|\right] + \frac{1}{2}\right].
		$$
	\end{pf}

	Denote by $b_\varepsilon$ de OSB of \eqref{eq:OSP_eps}.
	\begin{pr}($b_\varepsilon$ converges to $b$) \label{pr:b_eps->b} \\
		$\lim_{\varepsilon\downarrow0}b_\varepsilon(t) = b(t)$ for all $t\in[0, T)$.
	\end{pr}
	\begin{pf}
		We know that $V^\varepsilon(t, x)$ decreases in $\varepsilon$ and therefore $b_\varepsilon(t)$ increases as $\varepsilon$ gets higher. Hence, the function $b_0(t):= \lim_{\varepsilon\downarrow0}b_\varepsilon(t)$ is well defined, and since $A \leq b_\varepsilon(t) \leq b(t)$ for all $\varepsilon$, then $b(t) \leq b_0(t) \leq A$. On the other hand, recalling that $V^\varepsilon \rightarrow V$ as $\varepsilon\downarrow0$ (actually, it converges uniformly in compacts, see Proposition \ref{pr:V^eps->V}), then $V(t, b_\varepsilon(t)) \leq V^\varepsilon(t, b_\varepsilon(t)) + h_\varepsilon$, with $h_\varepsilon \downarrow 0$ as $\varepsilon\downarrow 0$, with, alongside the fact that $V^{\varepsilon}(t, b^\varepsilon(t)) = G(b_\varepsilon(t))$, gives us the relation $V(t, b_\varepsilon(t)) \leq G(b_\varepsilon) + h_\varepsilon$, from where we get $V(t, b_0(t)) \leq G(b_0(t))$ by taking $\varepsilon\downarrow 0$ (recall that $V$ is continous). Therefore $b_0(t) \leq b(t)$, and hece $b_0(t) = b(t)$.      
	\end{pf}
	
	
	
	\begin{pr}(Regularity of $V^\varepsilon$)\label{pr:V^eps_regularity}\\
		The value function $V^\varepsilon$ satisfies:
		\begin{enumerate}[label=(\roman{*}), ref=(\textit{\roman{*}})]
			\item \label{pr:V^eps_Lcontinuous} $V^\varepsilon$ is Lipschitz continuous on $[0, T_\varepsilon]\times\R$. 
			\item \label{pr:V^eps_C^12onC} $V^\varepsilon$ is $\mathcal{C}^{1,2}$ on $C^\varepsilon$ and on $D^\varepsilon$, and $\partial_tV^\varepsilon + \mathbb{L}_{X}V^\varepsilon = \lambda V^\varepsilon$ on $C^\varepsilon$. 
			\item \label{pr:V^eps_x-V^eps_t} The following relations hold for all $(t, x) \in [0, T)\times\R$ such that $x\neq b(t)$,
			\begin{align}
			\partial_t V^\varepsilon(t, x) &\leq \E\left[\left(-\partial_t X_{\tau^*}^{t, x}\right)^+\right], \label{eq:V^epst_up} \\		
			\partial_t V^\varepsilon(t, x) &\geq - \E\left[\left(-\partial_t X_{\tau^*}^{t, x}\right)^-\right] \nonumber \\
			&\ \ \ \ - \E\left[\mathbb{1}(\tau^* = T - t)\left(\alpha^2\frac{x + N^t - z}{\sinh(\alpha (T - t))}\right)^-\right], \label{eq:V^epst_low}
			\end{align}
			and
			\begin{align}\label{eq:V^eps_x}
			\partial_xV^\varepsilon(t, x) = -\E\left[e^{-\lambda\tau^*}\partial_x X_{\tau^*}^{t, x}\right],
			\end{align} 
			where $\tau^* = \tau^*(t, x)$ and\sidenote{Is it true?}
			\begin{align}\label{eq:N^eps^t}
			N^t = \sigma\int_{0}^{T - t} \frac{\sinh(\alpha (T - t))}{\sinh(\alpha (T - t - u))}\,\mathrm{d}W_u. 
			\end{align} 			
		\end{enumerate}
	\end{pr}
	
	\begin{pf}
		\ref{pr:V_Lcontinuous}\ \ \ Due to the martingale and super-martingale properties of $V^\varepsilon$, the fact that $V^\varepsilon \geq G$ on $[0, T)\times\R$ and $V^\varepsilon = G$ on $D^\varepsilon$, and inequality \eqref{eq:G_inequality}, one has the following for $0 < \delta \leq t$ and $\tau^* = \tau_\varepsilon^*(t, x)$, 
		\begin{align}
		V(t, x) - V(t - \delta, x) &\leq \E\left[V^\varepsilon(t + \tau^*, X_{\tau^*}^{t, x})\right] - \E_{t - \delta, x}\left[V^\varepsilon(t - \delta + \tau^*, X_{\tau^*}^{t - \delta, x})\right] \nonumber \\
		&\leq \E\left[G\left(X_{\tau^*}^{t, x}\right) - G\left(X_{\tau^*}^{t - \delta, x}\right)\right] \nonumber \\
		&\leq \E\left[\left(X_{\tau^*}^{t - \delta, x} - X_{\tau^*}^{t, x}\right)^+\right] \nonumber \\
		&= \delta\E\left[\left(-\partial_tX_{\tau^*}^{t_\delta, x}\right)^+\right] \label{eq:V^epst-V^epst-delta<} \\
		&\leq \delta L_{1, x, \varepsilon}, \nonumber
		\end{align}
		where the last equality holds for some $t_\delta\in(t - \delta, t)$ due to the mean value theorem, and
		$$
		L_{1, x, \varepsilon} = \sup_{0\leq t \leq T_\varepsilon}\E\left[\sup_{0\leq s \leq T_\varepsilon - t}|\partial_tX_s^{t, x}|\right].
		$$
		
		On the other hand, considering $0 < \delta \leq T_\varepsilon - t$, using similar arguments as the ones held in \eqref{eq:V^epst-V^epst-delta<}, and noticing that $\tau^*\wedge(T_\varepsilon - t - \delta)$ is admissible for $V^\varepsilon(t - \delta, x)$, we get
		\begin{align}
		V^\varepsilon(t + \delta, x) - V^\varepsilon(t, x) &\geq \E\left[V^\varepsilon\left(t + \delta + \tau^*\wedge(T_\varepsilon - t - \delta), X_{\tau^*\wedge(T_\varepsilon - t - \delta)}^{t + \delta, x}\right)\right] - \E\left[V^\varepsilon\left(t + \tau^*, X_{\tau^*}^{t, x}\right)\right] \nonumber \\
		&= \E\left[\mathbb{1}(\tau^* \leq T_\varepsilon - t - \delta)\left(V^\varepsilon\left(t + \delta + \tau^*, X_{\tau^*}^{t + \delta, x}\right) - V^\varepsilon\left(t + \tau^*, X_{\tau^*}^{t, x}\right)\right)\right] \nonumber \\
		&\ \ \ \ +\E\left[\mathbb{1}(\tau^* > T_\varepsilon - t - \delta)\left(V^\varepsilon\left(T_\varepsilon, X_{T_\varepsilon - t - \delta}^{t + \delta, x}\right) - V^\varepsilon\left(T_\varepsilon - \delta, X_{T_\varepsilon - t - \delta}^{t, x}\right)\right)\right] \nonumber \\
		&\geq \E\left[\mathbb{1}(\tau^*\leq T_\varepsilon - t - \delta)\left(G\left(X_{\tau^*\wedge(T_\varepsilon - t - \delta)}^{t + \delta, x}\right) - G\left(X_{\tau^*}^{t, x}\right)\right)\right] \nonumber \\
		&\ \ \ \ +\E\left[\mathbb{1}(\tau^* > T_\varepsilon - t - \delta)\left(G(X_{T_\varepsilon - t - \delta}^{t + \delta, x}) - V^{\varepsilon}(t + \tau^*, X_{\tau^*}^{t, x})\right)\right] \nonumber \\
		&\geq -\E\left[\mathbb{1}(\tau^*\leq T_\varepsilon - t - \delta)\left(X_{\tau^*}^{t, x} - X_{\tau^*}^{t + \delta, x}\right)^-\right] \nonumber \\
		&\ \ \ \ +\E\left[\mathbb{1}(\tau^* > T_\varepsilon - t - \delta)\left(G(X_{T_\varepsilon - t - \delta}^{t + \delta, x}) - V^{\varepsilon}(t + \tau^*, X_{\tau^*}^{t, x})\right)\right]. \label{eq:V^epst+delta-V^epst>}
		\end{align}
		Notice that
		$$
		\E\left[\mathbb{1}(\tau^*\leq T_\varepsilon - t - \delta)\left(X_{\tau^*}^{t, x} - X_{\tau^*}^{t + \delta, x}\right)^-\right] \leq \delta L_{1, x, \varepsilon}.
		$$
		Also, using the martingale property of $V^{\varepsilon}$,
		\begin{align}
		&\E\left[\mathbb{1}(\tau^* > T - T_\varepsilon - \delta)\left(G(X_{T_\varepsilon - t - \delta}^{t + \delta, x}) - V^{\varepsilon}(t + \tau^*, X_{\tau^*}^{t, x})\right)\right] \nonumber \\
		&= \E\left[\mathbb{1}(\tau^* > T - T_\varepsilon - \delta)\left(G(X_{T_\varepsilon - t - \delta}^{t + \delta, x}) - \E\left[V^{\varepsilon}(t + \tau^*, X_{\tau^*}^{t, x}) | \mathcal{F}_{T_\varepsilon - t - \delta}\right]\right)\right] \nonumber \\
		&= \E\left[\mathbb{1}(\tau^* > T - T_\varepsilon - \delta)\left(G(X_{T_\varepsilon - t - \delta}^{t + \delta, x}) - V^{\varepsilon}(T_\varepsilon - \delta, X_{T_\varepsilon - t - \delta}^{t, x})\right)\right]. \label{eq:V^eps_Lcont_1}
		\end{align}
		By definition of $V^\varepsilon$ and using the Itô-Tanaka formula, we derive the following,
		\begin{align}
		&V^{\varepsilon}(T_\varepsilon - \delta, X_{T_\varepsilon - t - \delta}^{t, x}) \nonumber \\
		&= \sup_{\tau \leq \delta}\E_{T_\varepsilon - \delta, X_{T_\varepsilon - t - \delta}^{t, x}}\left[e^{-\lambda\tau}G(X_{T_\varepsilon - \delta + \tau})\right] \nonumber \\
		&= G(X_{T_\varepsilon - t - \delta}^{t, x}) - \sup_{\tau \leq \delta}\E_{T_\varepsilon - \delta, X_{T_\varepsilon - t - \delta}^{t, x}}\left[\int_{0}^{\tau}e^{-\lambda u}\mathbb{1}\left(X_{T_\varepsilon - \delta + u} < A\right)\left(A - X_{T_\varepsilon - \delta + u} + \mu(T_\varepsilon - \delta + u, X_{T_\varepsilon - \delta + u})\right)\,\dif u\right. \nonumber \\
		&\hspace{6cm} \left. + \frac{1}{2}\int_{0}^{\tau}\mathbb{1}(X_{T_\varepsilon - \delta + u} = A)\,\dif l_u^A(X_{T_\varepsilon - \delta + \cdot}) \right] \nonumber \\
		&\leq G(X_{T_\varepsilon - t - \delta}^{t, x}) + \E_{T_\varepsilon - \delta, X_{T_\varepsilon - t - \delta}^{t, x}}\left[\int_{0}^{\delta}\left|A - X_{T_\varepsilon - \delta + u} + \mu(T_\varepsilon - \delta + u, X_{T_\varepsilon - \delta + u})\right|\,\dif u\right] + \frac{1}{2}l_\delta^A(X_{T_\varepsilon - \delta + \cdot}) \nonumber \\
		&\leq G(X_{T_\varepsilon - t - \delta}^{t, x}) + \E_{T_\varepsilon - \delta, X_{T_\varepsilon - t - \delta}^{t, x}}\left[\int_{0}^{\delta}M\left(1 + \left|X_{T_\varepsilon - \delta + u}\right|\right)\,\dif u\right] + \frac{\delta}{2} \nonumber \\
		&\leq G(X_{T_\varepsilon - t - \delta}^{t, x}) + \delta L_{2, x, \varepsilon}, \label{eq:V^eps_Lcont_2}
		\end{align}
		where we used the fact that $x \mapsto A - x + \mu(t, x)$ is Lipschitz continuous whenever $t\in[0, T_\varepsilon]$,
		$$
		L_{2, x, \varepsilon} = M\sup_{0\leq t\leq T_\varepsilon}\E_{T_\varepsilon - \delta, X_{T_\varepsilon - t - \delta}^{t, x}}\left[\sup_{t\leq s\leq T_\varepsilon}\left|X_{s}\right|\right] + \left(M + \frac{1}{2}\right),
		$$
		and $l_u^A(X)$ is the local time of the process $X$ at $A$ up to time $u$,
		$$
		l_u^A(X) = \lim_{h\downarrow 0} \frac{1}{h}\int_{0}^{u}\mathbb{1}(A - h \leq X_u \leq A + h)\,\dif \langle X, X\rangle_u.
		$$
		Plugging \eqref{eq:V^eps_Lcont_2} into \eqref{eq:V^eps_Lcont_1}, we get
		\begin{align}
		&\E\left[\mathbb{1}(\tau^* > T - T_\varepsilon - \delta)\left(G(X_{T_\varepsilon - t - \delta}^{t + \delta, x}) - V^{\varepsilon}(t + \tau^*, X_{\tau^*}^{t, x})\right)\right] \nonumber \\
		&\geq \E\left[\mathbb{1}(\tau^* > T - T_\varepsilon - \delta)\left(G(X_{T_\varepsilon - t - \delta}^{t + \delta, x}) - G(X_{T_\varepsilon - t - \delta}^{t, x})\right)\right] - \delta L_{2, x, \varepsilon} \nonumber \\ 
		&\geq - \E\left[\mathbb{1}(\tau^* > T - T_\varepsilon - \delta)\left(X_{T_\varepsilon - t - \delta}^{t, x} - X_{T_\varepsilon - t - \delta}^{t + \delta, x}\right)^-\right] - \delta L_{2, x, \varepsilon} \nonumber \\ 
		&\geq  \delta\left(L_{1, x, \varepsilon} - L_{2, x, \varepsilon}\right). \label{eq:V^eps_Lcont_3}
		\end{align}
		
		Now consider $\tau_\delta = \tau_\varepsilon^*(t - \delta, x)$ for $0 \leq \delta \leq t \leq T_\varepsilon$, and notice that $\tau_\delta\wedge(T_\varepsilon - t)$ is admissible for $V^{\varepsilon}(t, x)$. Then, argue as in \eqref{eq:V^epst+delta-V^epst>} to get
		\begin{align*}
		V^{\varepsilon}(t, x) - V^\varepsilon(t - \delta, x)
		&\geq -\E\left[\mathbb{1}(\tau_\delta\leq T_\varepsilon - t)\left(X_{\tau_\delta}^{t - \delta, x} - X_{\tau_\delta}^{t, x}\right)^-\right] \nonumber \\
		&\hspace{0.5cm} +\E\left[\mathbb{1}(\tau_\delta > T_\varepsilon - t)\left(G(X_{T_\varepsilon - t}^{t, x})  - V(t - \delta + \tau_\delta, X_{\tau_\delta}^{t - \delta, x})\right)\right],
		\end{align*}
		with
		$$
		\E\left[\mathbb{1}(\tau_\delta\leq T_\varepsilon - t)\left(X_{\tau_\delta}^{t - \delta, x} - X_{\tau_\delta}^{t, x}\right)^-\right] \leq \varepsilon L_{1, x, \delta},
		$$ 
		and
		\begin{align*}
		\E\left[\mathbb{1}(\tau_\delta > T_\varepsilon - t)\left(G(X_{T_\varepsilon - t}^{t, x})  - V(t - \delta + \tau_\delta, X_{\tau_\delta}^{t - \delta, x})\right)\right] &\geq  \delta\left(L_{1, x, \varepsilon} - L_{2, x, \varepsilon}\right).
		\end{align*}
		
		Additionally, for $0 \leq \delta \leq T_\varepsilon - t $ and $\tau^\delta = \tau_\varepsilon^*(t + \delta, x)$ one gets the following by proceeding as in \eqref{eq:V^epst-V^epst-delta<},
		\begin{align*}
		V^\varepsilon(t + \delta, x) - V^\varepsilon(t, x) &\leq  \delta L_{1, x, \delta},
		\end{align*}
				
		So far, as \sidenote{Trivial enough?} $L_{i, x, \delta} < \infty$ for $i =1, 2$, we have proved that, for any $\delta$ satisfying $|\delta| \leq t \leq T_\varepsilon$, there exists a constant $L_{x,\delta} < \infty$ such that
		$$
		|V^\varepsilon(t + \delta, x) - V^\varepsilon(t, x)| \leq \delta L_{x, \varepsilon},
		$$
		which means that $t\mapsto V^\varepsilon(t, x)$ is Lipschitz continuous in the interval $[0, T_\varepsilon]$. We will now prove that $V^\varepsilon$ is also Lipschitz continuous with respect to $x$ for all $t\in [0, T)$, which will completes the proof of \ref{pr:V^eps_Lcontinuous}.
		
		Since $G(x)$ decreases and $X_s^{t, x}$ increases as $x$ grows for all $s\in[0, T_\varepsilon - t]$, then $V^\varepsilon(t, x)$ decreases on $x$ for all $t\in[0, T_\varepsilon]$. Fix $(t, x)\in[0, T_\varepsilon]\times\R$ and $\delta > 0$. Consider $\tau^* = \tau_\varepsilon^*(t, x)$, and combine \eqref{eq:OSP}, \eqref{eq:OST}, \eqref{eq:G_inequality}, and \eqref{eq:OUB_rep_int}, to get 
		\begin{align}
		0 \geq V^\varepsilon(t, x + \delta) - V^\varepsilon(t, x) &\geq \E_{t, x + \delta}\left[e^{-\lambda\tau^*}G(X_{t + \tau^*})\right] - \E_{t, x}\left[e^{-\lambda\tau^*}G(X_{t + \tau^*})\right] \nonumber \\
		&= \E\left[e^{-\lambda\tau^*}\left(G(X_{\tau^*}^{t, x + \delta}) - G(X_{\tau^*}^{t, x})\right)\right] \nonumber\\
		&\geq -\E\left[e^{-\lambda\tau^*}\left(X_{\tau^*}^{t, x + \delta} - X_{\tau^*}^{t, x}\right)^+\right] \nonumber\\
		&= -\delta\E\left[e^{-\lambda\tau^*}\frac{\sinh(\alpha (T - t - \tau^*))}{\sinh(\alpha (T - t))}\right] \label{eq:V^epsx+delta-V^epsx}  \\
		&\geq -\delta \nonumber
		\end{align}
		Arguing similarly with $\tau_\delta = \tau_\varepsilon^*(t, x - \delta)$ yields
		\begin{align*}
		0 \geq V^\varepsilon(t, x) - V^\varepsilon(t, x - \delta) &\geq -\delta\E\left[e^{-\lambda\tau_\delta}\frac{\sinh(\alpha (T - t - \tau_\delta))}{\sinh(\alpha (T - t))}\right] \geq -\delta
		\end{align*}
		Then, $x\mapsto V^\varepsilon(t, x)$ is Lipschitz continuous for all $t\in[0, T)$, which alongside the Lipschitz continuity of $t\mapsto V^\varepsilon(t, x)$	in $[0, T - \varepsilon]$ for allows us to conclude that $V^\varepsilon$ is Lipschitz continuous on $[0, T - \varepsilon]\times\R$, this is, for any $0 < \varepsilon < T$, there exists $L_\varepsilon$ such that
		\begin{align*}
		|V^\varepsilon(t_1, x_1) - V^\varepsilon(t_2, x_2)| \leq L_\varepsilon(|t_1 - t_2| + |x_1 - x_2|),
		\end{align*} 
		for all $(t_1, x_1), (t_2, x_2) \in [0, T - \varepsilon]\times\R$. 	\vspace{0.3cm}
		
		\ref{pr:V^eps_C^12onC}\sidenote{This part is valid whenever $V^\varepsilon$ is continuous in $C^\varepsilon$ and both $\mu$ and $\sigma$ are locally $\alpha$-Hölder continuous in $[0, T)\times\R$}\ \ The fact that $\partial_tV^\varepsilon + \mathbb{L}_{X}V^\varepsilon = \lambda V^\varepsilon$  on $C^\varepsilon$ comes right after the strong Markov property of $(t, X_t)$ (See \citet[Section 7.1]{goran-optimal} for more details).
		
		Also, since $V^\varepsilon$ is continuous in $C^\varepsilon$ (see \ref{pr:V^eps_Lcontinuous}), $\mu$ in $\eqref{eq:OUB_drift}$ is locally $\alpha$-Hölder continuous in $[0, T_\varepsilon]\times\R$, and $\sigma$ is constant, then, one can borrow standard result from parabolic PDE's theory (see \citet[Section 3, Theorem 9]{friedman1964partial}) to guarantee that, for an open rectangle $R\subset C^\varepsilon$, the first initial-boundary value problem
		\begin{align}
		\partial_t f + \mathbb{L}_{X} f - \lambda f &= 0 && \text{in } R, \label{eq:PDE1} \\
		f &= V^\varepsilon && \text{on } \partial R, \label{eq:PDE2}
		\end{align}
		has an unique solution $f$ such that $f\in \mathcal{C}^{1, 2}(R)$. Therefore, we can use Itô's formula on $f(X_{t + s})$ at $s = \tau_{R^c}$, this is, the first time $X_{t + s}$ exits $R$, and then take $\Pr_{t, x}$-expectation with $x \in R$, which guarantees the vanishing of the martingale term and yields, alongside both (\ref{eq:PDE1}) and (\ref{eq:PDE2}), the equality $\E_{t, x}[V^\varepsilon(X_{t + \tau_{R^c}})] = f(t, x)$. Finally, notice that, due to the strong Markov property, $\E_{t, x}[V^\varepsilon(X_{t + \tau_{R^c}})] = V^\varepsilon(t, x)$.
		\vspace{0.3cm}
		
		\ref{pr:V^eps_x-V^eps_t}\ \ As we have just proved in \ref{pr:V^eps_C^12onC}, the partial derivatives of $V^\varepsilon$ exist on $C^\varepsilon$, and trivially they also exist on $\left(D^\varepsilon\right)^{\circ}$ since $V^\varepsilon = G$ on $D^\varepsilon$. Therefore, for any $(t, x)\in [0, T)\times \R$ with $x\neq b_\varepsilon(t)$, we can take $\delta\rightarrow 0$ in \eqref{eq:V^epst-V^epst-delta<}, \eqref{eq:V^epst+delta-V^epst>}, and \eqref{eq:V^epsx+delta-V^epsx}, to derive equations \eqref{eq:V^epst_up}, \eqref{eq:V^epst_low}, and \eqref{eq:V^eps_x} respectively.	
	\end{pf}
	
	
	\begin{pr}\label{pr:OSB_LC_OUB}
		The OSB $b$ is Lipschitz continuous on $[0, T_\varepsilon]$ for any $0 < \varepsilon < T$.
	\end{pr}
	
	\begin{pf}
		Consider the function $W(t, x) = V^\varepsilon(t, x) - G(x)$ and the interval. Propositions \ref{pr:boundary_existence} and \ref{pr:V^eps_regularity} guarantee that $W$ is continuous and $W(t, A) > 0$ for all $t\in [0, T_\varepsilon]$. Hence, there exists $a > 0$ such that $W(t, A) \geq a$ for all $t\in [0, T_\varepsilon]$. Therefore, for all $\delta$ such that $0 < \delta \leq a$, the equation $W(t, x) = \delta$ has solution on $C^\varepsilon$ for all $t\in [0, T_\varepsilon]$. Moreover, since Proposition \ref{pr:V^eps_regularity} states that $\partial_x W > 0$ on $C^\varepsilon$, this solution is unique for each $t$ and we can denote it by $b_{\delta, \varepsilon}(t)$, where  $b_{\delta, \varepsilon}:[0, T_\varepsilon]\rightarrow (b_\varepsilon(t), A]$. The continuity of $b_{\delta, \varepsilon}$ on $[0, T_\varepsilon]$ comes after the continuity of $W$. Furthermore, the implicit function theorem guarantees that $b_{\delta, \varepsilon}$ is differentiable and  
		\begin{align}\label{eq:b_delta'}
		b_{\delta, \varepsilon}'(t) = -\partial_t W(t, b_{\delta, \varepsilon}(t)) / \partial_x W(t, b_\delta(t)).
		\end{align}
		
		Notice that $b_{\delta, \varepsilon}$ is decreasing in $\delta$ and therefore it converges point-wise to some limit function $b_0(t)$, which satisfies $b_0 > b_\varepsilon$ in $[0, T_\varepsilon]$ as $b_\delta > b$ for all $\delta$. On the other hand, since $W(t, b_{\delta, \varepsilon}(t)) = \delta$ and $W$ is continuous it follows that $W(t, b_0(t)) = 0$ after taking $\delta\rightarrow 0$, which means that $b_0 < b_\varepsilon$ in $[0, T_\varepsilon]$ and hence $b_0 = b_\varepsilon$ in $[0, T_\varepsilon]$.
		
		%%%
		Combining \eqref{eq:b_delta'}, \eqref{eq:V^epst_up} and \eqref{eq:V^eps_x}, then using the explicit form \eqref{eq:OUB_t_rep_int}, denoting $x_\delta = b_{\delta, \varepsilon}(t)$ and $\tau_\delta = \tau_\varepsilon^*(t, x_\delta)$, and noticing that $x_\delta < A$ for all $\delta$, one gets\sidenote{We will assume $\alpha > 0$ as its signs is irrelevant for the underlying OUB. See \eqref{eq:OUB_rep_int}.}
		\begin{align}
		b_{\delta, \varepsilon}'(t) &\geq -\displaystyle{\frac{\E\left[\left(-\partial_t X_{\tau_\delta}^{t, x_\delta}\right)^+\right]}{1 - \E\left[e^{-\lambda\tau_\delta}\frac{\sinh(\alpha (T - t - \tau_\delta))}{\sinh(\alpha (T - t))}\right]}} \nonumber \\
		&\geq -\frac{\E\left[\left(-\partial_t X_{\tau_\delta}^{t, x_\delta}\right)^+\right]}{\E\left[\frac{\sinh(\alpha(T - t)) - \sinh(\alpha (T - t - \tau_\delta))}{\sinh(\alpha (T - t))}\right]} \nonumber \\
		&= -\frac{\E\left[\left(\alpha\frac{\sinh(\alpha \tau_\delta)(x_\delta - z\cosh(\alpha(T - t)))}{\sinh(\alpha(T - t))} + \alpha\sinh(\alpha(T - t))M_{\tau_\delta}^t\right)^+\right]}{\E\left[\sinh(\alpha(T - t)) - \sinh(\alpha (T - t - \tau_\delta))\right]} \nonumber \\
		&\geq -\alpha\frac{(A + |z|\cosh(\alpha T))}{\sinh(\alpha\varepsilon)}\frac{\E\left[\sinh(\alpha \tau_\delta)\right]}{\E\left[\sinh(\alpha(T - t)) - \sinh(\alpha (T - t - \tau_\delta))\right]} \nonumber \\
		&\ \ \ \ -\alpha\sinh(\alpha T)\frac{\E\left[\left(M_{\tau_\delta}^t\right)^+\right]}{\E\left[\sinh(\alpha(T - t)) - \sinh(\alpha (T - t - \tau_\delta))\right]} \nonumber
		\end{align}
		%		&= -\frac{\E\left[\left(\alpha^2\frac{\tau_\delta\cosh(\zeta_{1,\delta})(A + |z|\cosh(\alpha T))}{\sinh(\alpha(T - t))} + \alpha\sinh(\alpha(T - t))M_{\tau_\delta}^t\right)^+\right]}{\alpha\E\left[\tau_\delta\cosh(\zeta_{2,\delta})\right]} \nonumber \\
		%		&\geq -\frac{\E\left[\left(\alpha^2\frac{\tau_\delta\cosh(\alpha T)(A + |z|\cosh(\alpha T))}{\sinh(\alpha \varepsilon)} + \alpha\sinh(\alpha T) M_{\tau_\delta}^t\right)^+\right]}{\alpha\E\left[\tau_\delta\right]} \nonumber \\
		%		&\geq -L_{1,\varepsilon} \left(L_{1} + \frac{\E\left[\left(M_{\tau_\delta}^t \right)^+\right]}{\E\left[\tau_\delta\right]}\right) \label{eq:b_delta'>}
		%		\end{align}
		where $M_s^t$ is given in \eqref{eq:Ms}.
		
		We can now apply the mean value theorem to the function $\sinh(t)$ to come up with random variables $\zeta_{1,\delta} \in (0, \alpha \tau_\delta)$ and $\zeta_{2,\delta} \in (\alpha(T - t - \tau_\delta), \alpha(T - t))$, such that
		\begin{align}\label{eq:sinh_bound}
		\frac{\E\left[\sinh(\alpha \tau_\delta)\right]}{\E\left[\sinh(\alpha(T - t)) - \sinh(\alpha (T - t - \tau_\delta))\right]} &= \frac{\E\left[\alpha\tau_\delta\cosh(\zeta_{1,\delta})\right]}{\E\left[\alpha\tau_\delta \cosh(\zeta_{2,\delta})\right]} \nonumber \\ 
		&\leq \cosh(\alpha T).
		\end{align}
		Also notice that, according to \eqref{eq:Ms}, $M_{\tau_\delta}^t$ admits the representation
		\begin{align*}
		M_s^{\tau_\delta} &\leq \sinh(\alpha\tau_\delta)\int_{0}^{\tau_\delta}\frac{\cosh(\alpha u)}{\sinh^2(\alpha(T - t - u))}\,\dif W_u \\
		&\ \ \ \ + \cosh(\alpha\tau_\delta)\int_{0}^{\tau_\delta}\frac{\sinh(\alpha u)}{\sinh^2(\alpha(T - t - u))}\,\dif W_u,
		\end{align*}
		which can be used, alongside \eqref{eq:sinh_bound}, to derive the following bound
		\begin{align}
		&\frac{\E\left[\left(M_{\tau_\delta}^t\right)^+\right]}{\E\left[\sinh(\alpha(T - t)) - \sinh(\alpha (T - t - \tau_\delta))\right]} \nonumber \\ 
		&\leq \cosh(\alpha T)\frac{\E\left[\sinh(\alpha\tau_\delta)\left(\int_{0}^{\tau_\delta}\frac{\cosh(\alpha u)}{\sinh^2(\alpha(T - t - u))}\,\dif W_u\right)^+\right]}{\E\left[\sinh(\alpha\tau_\delta)\right]} \label{eq:term1} \\
		&\ \ \ \ +\cosh(\alpha T)\frac{\E\left[\cosh(\alpha\tau_\delta)\left(\int_{0}^{\tau_\delta}\frac{\sinh(\alpha u)}{\sinh^2(\alpha(T - t - u))}\,\dif W_u\right)^+\right]}{\E\left[\sinh(\alpha\tau_\delta)\right]}. \label{eq:term2}
		\end{align}
		We can find an upper bound for term \eqref{eq:term1} by using the fact that $(x)^+ \leq 1 + x^2$, for all $x\in \R$, as follows,
		\begin{align*}
		&\frac{\E\left[\sinh(\alpha\tau_\delta)\left(\int_{0}^{\tau_\delta}\frac{\cosh(\alpha u)}{\sinh^2(\alpha(T - t - u))}\,\dif W_u\right)^+\right]}{\E\left[\sinh(\alpha\tau_\delta)\right]} \\
		&\leq \frac{\E\left[\sinh(\alpha\tau_\delta)\left(1 + \sup_{s\leq \tau_\delta}\left(\int_{0}^{s}\frac{\cosh(\alpha u)}{\sinh^2(\alpha(T - t - u))}\,\dif W_u\right)^2\right)\right]}{\E\left[\sinh(\alpha\tau_\delta)\right]} \\
		&\leq 1 + \sinh(\alpha T)\frac{\E\left[\sup_{s\leq \tau_\delta}\left(\int_{0}^{s}\frac{\cosh(\alpha u)}{\sinh^2(\alpha(T - t - u))}\,\dif W_u\right)^2\right]}{\E\left[\sinh(\alpha\tau_\delta)\right]} \\
		&\leq 1 + K_1\frac{\E\left[\int_{0}^{\tau_\delta}\frac{\cosh^2(\alpha u)}{\sinh^4(\alpha(T - t - u))}\,\dif u\right]}{\E\left[\sinh(\alpha\tau_\delta)\right]} \\
		&\leq 1 + K_1\frac{\E\left[\tau_\delta\frac{\cosh^2(\alpha T)}{\sinh^4(\alpha(\varepsilon))}\right]}{\E\left[\sinh(\alpha\tau_\delta)\right]} \\
		&= 1 + K_1\frac{\E\left[\tau_\delta\frac{\cosh^2(\alpha T)}{\sinh^4(\alpha(\varepsilon))}\right]}{\E\left[\alpha\tau_\delta\cosh(\zeta_{1,\delta})\right]} \\
		&\leq 1 + K_1\frac{\cosh^2(\alpha T)}{\alpha\sinh^4(\alpha(\varepsilon))}.
		\end{align*}
		From some positive constant $K_1$ coming after applying the Burkholder-Davis-Gundy inequality (see \citet[Corollary 4.2]{revuz_continuous_1999}).
		
		Similar arguments yield an upper bound for term \eqref{eq:term2},
		\begin{align*}
		&\frac{\E\left[\cosh(\alpha\tau_\delta)\left(\int_{0}^{\tau_\delta}\frac{\sinh(\alpha u)}{\sinh^2(\alpha(T - t - u))}\,\dif W_u\right)^+\right]}{\E\left[\sinh(\alpha\tau_\delta)\right]} \\
		&\leq \cosh(\alpha T)\frac{\E\left[\sup_{s\leq \tau_\delta}\left|\int_{0}^{s}\frac{\sinh(\alpha u)}{\sinh^2(\alpha(T - t - u))}\,\dif W_u\right|\right]}{\E\left[\sinh(\alpha\tau_\delta)\right]} \\
		&\leq K_2\cosh(\alpha T)\frac{\E\left[\left(\int_{0}^{\tau_\delta}\frac{\sinh^2(\alpha u)}{\sinh^4(\alpha(T - t - u))}\,\dif u\right)^{1/2}\right]}{\E\left[\sinh(\alpha\tau_\delta)\right]} \\
		&\leq K_2\frac{\cosh(\alpha T)}{\sinh^2(\alpha\varepsilon)}\frac{\E\left[\sqrt{\tau_\delta}\sinh(\alpha \tau_\delta)\right]}{\E\left[\sinh(\alpha\tau_\delta)\right]} \\
		&\leq K_2\sqrt{T}\frac{\cosh(\alpha T)}{\sinh^2(\alpha\varepsilon)}
		\end{align*}
		for a positive constant $K_2$.
		
		
		
		
		
		
		
		
		
		
		
		
		%		
		%		Notice that, since $x^+ \leq 1 + x^2$ for all $x\in \R$, then
		%		\begin{align*}
		%			\E\left[\left(M_{\tau_\delta}^t\right)^+\right] &\leq 1 + \E\left[\left(M_{\tau_\delta}^t\right)^2\right] \\
		%			&\leq 1 + \E\left[\sup_{s\leq \tau_\delta}\left(M_s^t\right)^2\right] \\
		%			&\leq L_\varepsilon\left(1 + \E\left[\int_0^{\tau_\delta} \frac{\sinh^{2}(\alpha(\tau_\delta - s))}{\sinh^{4}(\alpha(T - t - s))} \dif s\right]\right),
		%		\end{align*}
		%		where the last inequality comes after the Burkholder-Davis-Gundy inequality (see \citet[Corollary 4.2]{revuz_continuous_1999})
		
		On the other hand, using now \eqref{eq:Vt_low} and following similar arguments we get
		%		\begin{align}
		%		b_\delta'(t) &\leq \displaystyle{\frac{\E\left[\left(-\partial_t X_{\tau_\delta}^{t, x_\delta}\right)^-\right]}{1 - \E\left[e^{-\lambda\tau_\delta}\frac{\sinh(\alpha (T - t - \tau_\delta))}{\sinh(\alpha (T - t))}\right]}} \nonumber \\
		%		&\leq \frac{\E\left[\left(-\partial_t X_{\tau_\delta}^{t, x_\delta}\right)^-\right]}{\E\left[\frac{\sinh(\alpha(T - t)) - \sinh(\alpha (T - t - \tau_\delta))}{\sinh(\alpha (T - t))}\right]} \nonumber \\
		%		&= \frac{\E\left[\left(-\alpha\sinh(\alpha \tau_\delta)(z\cosh(\alpha(T - t)) - x_\delta - M_{\tau_\delta})\right)^-\right]}{\sinh(\alpha (T - t))\E\left[\sinh(\alpha(T - t)) - \sinh(\alpha (T - t - \tau_\delta))\right]} \nonumber \\
		%		&\leq \frac{\E\left[\alpha\sinh(\alpha \tau_\delta)\left(|z|\cosh(\alpha(T - s)) + A + \sup_{0\leq s\leq T}M_{s}\right)^-\right]}{\sinh(\alpha (T - t))\E\left[\sinh(\alpha(T - t)) - \sinh(\alpha (T - t - \tau_\delta))\right]} \nonumber \\
		%		&= \frac{\E\left[\alpha^2\tau_\delta\cosh(\zeta_{1,\delta})\left(|z|\cosh(\alpha(T - s)) + A + \sup_{0\leq s\leq T}M_{s}\right)^-\right]}{\alpha\sinh(\alpha (T - t))\E\left[\tau_\delta\cosh(\zeta_{2,\delta})\right]} \nonumber \\
		%		&\leq \frac{\E\left[\alpha^2\tau_\delta\cosh(\alpha T)\left(|z|\cosh(\alpha(T - s)) + A + \sup_{0\leq s\leq T}M_{s}\}\right)^-\right]}{\alpha\sinh(\alpha (T - t))\E\left[\tau_\delta\right]} \nonumber
		%		\end{align}
		\begin{align*}
		b_\delta'(t) &\leq \displaystyle{\frac{\E\left[\left(-\partial_t X_{\tau_\delta}^{t, x_\delta}\right)^-\right]}{1 - \E\left[e^{-\lambda\tau_\delta}\frac{\sinh(\alpha (T - t - \tau_\delta))}{\sinh(\alpha (T - t))}\right]}} + \frac{\E\left[\mathbb{1}(\tau^* = T - t)\left(\alpha^2\frac{x + N^t - z}{\sinh(\alpha (T - t))}\right)^-\right]}{1 - \E\left[e^{-\lambda\tau_\delta}\frac{\sinh(\alpha (T - t - \tau_\delta))}{\sinh(\alpha (T - t))}\right]},
		\end{align*}
		with
		\begin{align*}
		\displaystyle{\frac{\E\left[\left(-\partial_t X_{\tau_\delta}^{t, x_\delta}\right)^-\right]}{1 - \E\left[e^{-\lambda\tau_\delta}\frac{\sinh(\alpha (T - t - \tau_\delta))}{\sinh(\alpha (T - t))}\right]}}&\leq \frac{\E\left[\alpha^2\tau_\delta\cosh(\alpha T)\left(\inf_{0\leq s\leq T}b(s) + \sigma M_{\tau_\delta}^t - |z|\cosh(\alpha T )\right)^-\right]}{\alpha\sinh(\alpha\varepsilon)\E\left[\tau_\delta\right]},
		\end{align*}
		and
		\begin{align*}
		\frac{\E\left[\mathbb{1}(\tau^* = T - t)\left(\alpha^2\frac{x + N^t - z}{\sinh(\alpha (T - t))}\right)^-\right]}{1 - \E\left[e^{-\lambda\tau_\delta}\frac{\sinh(\alpha (T - t - \tau_\delta))}{\sinh(\alpha (T - t))}\right]} &\leq \frac{\E\left[\mathbb{1}(\tau^* = T - t)\alpha\left(\alpha(x + N^t - z)\right)^-\right]}{\E\left[\sinh(\alpha(T - t)) - \sinh(\alpha (T - t - \tau_\delta))\right]} \\
		&= \frac{\E\left[\mathbb{1}(\tau^* = T - t)\left(\alpha(x + N^t - z)\right)^-\right]}{\E\left[\tau_\delta\cosh(\zeta_{2,\delta})\right]} \\
		&\leq \frac{\E\left[\mathbb{1}(\tau^* = T - t)\left(\alpha(\inf_{0\leq s\leq T}b(s) + \inf_{0\leq t\leq T}N^t - |z|)\right)^-\right]}{\varepsilon\E\left[\mathbb{1}(\tau^* = T-t)\right]}.
		\end{align*}
	\end{pf} 	
	
	
	\begin{pr}(Smooth fit condition)\label{pr:smooth.fit}
		The smooth fit condition holds, i.e., $\partial_xV(t,b(t)) = -1$ for all $t\in[0,T)$. 
	\end{pr}
	
	\begin{pf}
		\sidenote{\textcolor{red}{OUB:} $\partial_x^+V(t, b(t)) \leq -1$ relies on the integral representation (\ref{eq:OUB_rep_int}) of the OUB.}\ \ Take a point $(t, b(t))$ for some $t\in[0, T)$ and consider $\varepsilon > 0$. Since $(t, b(t)) \in D$ and $(t, b(t) + \varepsilon) \in C$, we get that $\varepsilon^{-1}(V(t, b(t) + \varepsilon) - V(t, b(t))) \geq \varepsilon^{-1}(G(b(t) + \varepsilon) - G(b(t))) = -1$. Therefore $\partial_x^+V(t, b(t)) \geq -1$. 
		
		On the other hand, setting $x = b(t)$ in \eqref{eq:Vx+eps-Vx} we get
		\begin{align*}
		\varepsilon^{-1}(V(t, b(t) + \varepsilon) - V(t, b(t))) \leq -\E\left[e^{-\lambda\tau_\varepsilon}\frac{\sinh(\alpha (T - t - \tau_\varepsilon))}{\sinh(\alpha (T - t))}\right],
		\end{align*}
		where $\tau_\varepsilon = \tau^*(t, b(t) + \varepsilon)$. Since the expression inside the expectation is lower than $1$ for all $\varepsilon$, we can use the dominated convergence theorem in the above inequality to get
		\begin{align}
		\partial_x^+V(t, b(t)) \leq -\E\left[e^{-\lambda\tau_0}\frac{\sinh(\alpha (T - t - \tau_0))}{\sinh(\alpha (T - t))}\right],
		\end{align}
		with\sidenote{$\tau_0$ exists and it is greater than $\tau^*(t, b(t))$ since the sequence ${\tau_\varepsilon}$ is decreasing w.r.t. $\varepsilon$} $\tau_0 := \lim_{\varepsilon\rightarrow0} \tau_\varepsilon$. Since $b$ is continuous and piecewise monotone ($b$ is proven to be Lipschitz continuous in Proposition \ref{pr:OSB_LC_OUB}) we can guarantee that $\tau_0 = \tau^{*}(t, b(t)) = 0\ \Pr$-a.s. (see \citet{cox_embedding_2015}[Corollary 8]), and therefore $\partial_x^+V(t, b(t)) \leq -1$.
		
		Finally, the smooth fit condition comes after realizing that $\partial_x^-V(t, b(t)) = -1$ since $V = G$ on $D$.   
	\end{pf}
	
	\section{The early exercise premium and the free-boundary equation}
	
	Propositions \ref{pr:boundary_existence}, \ref{pr:V_regularity}, \ref{pr:OSB_LC_OUB}, and \ref{pr:smooth.fit}, allow us to apply an extension of the Itô's formula (see \citet{d2020discounted}[Lemma A2]) to $e^{-\lambda s}V(t + s, X_{t + s})$, which, after setting $s = T - t$, taking $\Pr_{t, x}$-expectation, and shifting the integrating variable $t$ units, yields
	\begin{align}\label{eq:V_after_Ito}
	V(t, x) = e^{-\lambda (T - t)}G(z) - \int_{t}^{T}\E_{t, x}\left[e^{-\lambda (u - t)}(\partial_t V + \mathbb{L} V - \lambda V)(u,\ X_u)\right]\,\mathrm{d}u,
	\end{align} 
	where the martingale term is vanished after taking $\Pr_{t, x}$-expectation and the local time term does not appear due to the smooth fit condition. Recall that $\partial_t V + \mathbb{L} V = \lambda V$ on $C$ and $V = G$ on $D$. Therefore, (\ref{eq:V_after_Ito}) turns into the pricing formula
	\begin{equation}\label{eq:early_excersice_premium}
	\begin{aligned}
	&V\left(t, x\right) \\
	&= e^{-\lambda (T - t)}(A - z)^+ \\
	&\ \ \ \ - \int_{t}^{T}\E_{t, x}\left[e^{-\lambda (u - t)}\left(\left(\alpha\coth(\alpha(T - u)) + \lambda\right)\ X_{u} - \frac{z\alpha}{\sinh(\alpha(T - u))} - \lambda A \right)\mathbb{1}\left(X_{u}\leq b(u)\right)\right]\,\mathrm{d}u.
	\end{aligned}
	\end{equation}
	%The first term of the right hand sum stands for the price of the European put option written on the same asset and expiring on the same date. The second term, called the \textit{early exercise premium}, represents the cost of having the added flexibility of exercising any time before $T$.  
	Equation (\ref{eq:early_excersice_premium}) can be simplified by using $\E_{t, x}[X_u\mathbb{1}(X_u\leq a)] = \E_{t, x}[X_u|X_u\leq a]\Pr_{t, x}(X_u \leq a)$ and noticing that, since $X_u\sim\mathcal{N}(\nu_u(t, x), \gamma_u^2(t)))$ under $\Pr_{t, x}$, with
	\begin{align}
	\nu_u(t, x) &= \frac{x\sinh(\alpha (T - u)) + z\sinh(\alpha (u - t))}{\sinh(\alpha (T - t))}, \label{eq:mean_Xu}\\
	\gamma_u^2(t) &= \frac{\sigma^2}{\alpha}\frac{\sinh(\alpha (u - t))\sinh(\alpha (T - u))}{\sinh(\alpha (T - t))} \label{eq:var_Xu}.
	\end{align}
	Then $\E_{t, x}[X_u | X_u \leq a] = \nu_u(t, x) - \gamma_u(t)\frac{\phi(\hat{a})}{\Phi(\hat{a})}$, where $\phi$ and $\Phi$ are the density and distribution of a standard normal random variable, and $\hat{a} = \frac{a - \nu_u(t, x)}{\gamma_u(t)}$. We thereby end up with the pricing formula
	\begin{align}\label{eq:pricing_formula}
	V(t, x) = e^{-\lambda (T - t)}(A - z)^+ - \int_{t}^{T}e^{-\lambda (u - t)}K_{\alpha, \sigma, \lambda}(t, x, u, b(u))\,\mathrm{d}u,
	\end{align}
	where
	\begin{align}\label{eq:kernel}
	&K_{\alpha, \sigma, \lambda}(t_1, x_1, t_2, x_2) \nonumber \\
	&=\left(\alpha\coth(\alpha(T - t_2)) + \lambda\right)\left(\nu_{t_2}(t_1, x_1)\Phi\left(y_{\alpha, \sigma}(t_1, x_1, t_2, x_2)\right) - \gamma_{t_2}(t_1)\phi(y_{\alpha, \sigma}(t_1, x_1, t_2, x_2))\right) \nonumber \\
	&\ \ \ \ - \left(\frac{z\alpha}{\sinh(\alpha(T - t_2))} + \lambda A\right)\Phi\left(y_{\alpha, \sigma}(t_1, x_1, t_2, x_2)\right),
	\end{align}
	with
	\begin{align*}
	y_{\alpha, \sigma}(t_1, x_1, t_2, x_2) = \frac{x_2 - \nu_{t_2}(t_1, x_1)}{\gamma_{t_2}(t_1)}.
	\end{align*}
	Finally, by taking $x \uparrow b(t)$ in (\ref{eq:pricing_formula}) we derive the free-boundary equation
	\begin{align}\label{eq:free-boundary_equation}
	b(t) = A - e^{-\lambda (T - t)}(A - z)^+ + \int_{t}^{T}e^{-\lambda (u - t)}K_{\alpha, \sigma, \lambda}(t, b(t), u, b(u))\,\mathrm{d}u.
	\end{align}
	
	\if\uniqtheo1
	\begin{thm}
		The integral equation (\ref{eq:free-boundary_equation}) has unique solution among the class of continuous functions, $c:[0, T]\mapsto (-\infty, A]$, of bounded variation.
	\end{thm}
	
	\begin{pf}
		% Initializations and x -> V^c(t,x) is twice continuously differentiable
		
		Assume we have a function $c:[0,T]\rightarrow \mathbb{R}$ that solves \eqref{eq:free-boundary_equation} and define $V^c$ as in \eqref{eq:pricing_formula} but with $b$ replaced by $c$. It turns out that the integrand in \eqref{eq:pricing_formula} is twice continuously differentiable with respect to $x$, and therefore we can obtain $\partial_x V^c$ and $\partial_{xx} V^c$, and guarantee their continuity on $[0, T]\times R$, by differentiating inside the integral symbol.
		
		% Infinitesimal of X acting on V^c
		
		It is easy to check that $\partial_t V^c(t, x) + \mathbb{L}V^c(t, x) = \lambda V^c(t, x) - \left(\left(\alpha\coth(\alpha(T - u)) + \lambda\right)\ x - \frac{z\alpha}{\sinh(\alpha(T - t))} - \lambda A \right)\mathbb{1}\left(x\leq c(t)\right)$, which, alongside the fact that $V^c$, $\partial_x V^c$ and $\partial_{xx} V^c$ are continuous on $[0, T]\times \R$, we get the continuity of $\partial_t V^c$ on $C_1 \cup C_2$, where
		\begin{align*}
		C_{1} := \{(t,x) \in [0,T) \times \mathbb{R} : x > c(t)\},\ \ 
		C_{2} := \{(t,x) \in [0,T) \times \mathbb{R} : x < c(t)\}.
		\end{align*}
		
		% Itô's Formula for F
		
		Define the function $F^{(t)}(s, x) := e^{-\lambda s}V^{c}(t + s, x)$ with $s\in[0, T-t)$, $x\in\mathbb{R}$, and consider %the sets
		\begin{align*}
		C_{1}^t := \{(s,x) \in C_1 : t \leq s < T \},\ \ 
		C_{2}^t := \{(s,x) \in C_2 : t \leq s < T \}.
		\end{align*}
		
		Notice that $F^{(t)}$ satisfies the hypothesis of \citet{d2020discounted}[Lemma A2] with $C = C_1^t$ and $D^{\circ} = C_2^t$: $F^{(t)}$, $\partial_xF^{(t)}$, and $\partial_{x^2}F^{(t)}$ are continuous on $[0, T)\R$; it has been proved that $F^{(t)}$ is $\mathcal{C}^{1,2}$ on $C_1^t$ and $C_2^t$; we are assuming that $c$ is a continuous function of bounded variation; and $(\partial_tF^{(t)} + \mathbb{L}_X F^{(t)})(s, x) = -e^{-\lambda s}\left(\left(\alpha\coth(\alpha(T - t - s)) + \lambda\right)\ x - \frac{z\alpha}{\sinh(\alpha(T - t - s))} - \lambda A \right)\mathbb{1}\left(x\leq c(t - s)\right)$ is locally bounded on $C_1^t \cup C_2^t$.
		
		Thereby, we can use \citet{d2020discounted}[Lemma A2] to obtain the following change-of-variable formula, which is missing the local time term because of the continuity of $F_x$ on $[0, T)\times\mathbb{R}$:
		\begin{align}
		e^{-\lambda s}&V^c(t + s,\ X_{t + s})\nonumber\\
		&= V^c(t, x) - \int_{t}^{t + s}e^{-\lambda(u - t)}\left(\left(\alpha\coth(\alpha(T - u)) + \lambda\right)\ X_{u} - \frac{z\alpha}{\sinh(\alpha(T - u))} - \lambda A \right)\mathbb{1}\left(X_{u}\leq b(u)\right)\,\mathrm{d}u + M_{s}^{(1)}, \label{eq:V^c_changed}
		\end{align}
		with $M_{s}^{(1)} = \int_{t}^{t + s}e^{-\lambda (u - t)}\sigma \partial_xV^{c}(u, X_{u})\,\mathrm{d}B_{u}$. Notice that $(M_{s}^{(1)})_{s = 0}^{T - t}$ is a martingale under $\mathbb{P}_{t, x}$.
		
		% Ito's formula for G
		
		In the same way, we can apply \citet{d2020discounted}[Lemma A2], using now the function $F(s, x) = e^{-\lambda s}G(X_{t + s})$, and taking $C = \{(s, x) \in [0, T - t)\times\mathbb{R} : x > A\}$ and $D^{\circ} = \{(s, x) \in [0, T - t)\times\mathbb{R} :$ $ x < A\}$, thereby getting
		\begin{align}\label{eq:G_changed}
		e^{-\lambda s}G(X_{t+s}) &= G(x) - \int_{t}^{t + s}e^{-\lambda(u - t)}\left(\left(\alpha\coth(\alpha(T - u)) + \lambda\right)\ X_{u} - \frac{z\alpha}{\sinh(\alpha(T - u))} - \lambda A \right)\mathbb{1}\left(X_{u}\leq A\right)\,\mathrm{d}u \\
		&\ \ \ - M_{s}^{(2)} + \frac{1}{2}\int_t^{t + s}e^{-\lambda (u - t)}\mathbbm{1}(X_u = S)\,\mathrm{d}l_s^S(X), \nonumber
		\end{align}
		where $M_{s}^{(2)} = \sigma\int_{t}^{t + s}e^{-\lambda (u - t)}\mathbbm{1}(X_u < S)\,\mathrm{d}B_{u}$ is a martingale under $\mathbb{P}_{t, x}$, and $l_s^S(X)$ is the local time of $X$ at $A$ up to time $s$.
		
		% V^c = G on C_2!!
		
		Consider the following stopping time,
		\begin{align}\label{eq:stopping_rho}
		\rho_c := \inf\left\{0 \leq s\leq T - t: X_{t + s} \geq c(t + s)\right\}.
		\end{align}
		Fix $(t, x)$ such that $x \leq c(t)$. As assuming $c(t) < A$ for all $t \in (0, T)$, we can ensure, for  that $\mathbbm{1}(X_{t + s} \leq c(t + s)) = \mathbbm{1}(X_{t + s} \leq A) = 1$ for all $s \in [0, \rho_c)$, as well as $\int_t^{t + s}e^{-\lambda (u - t)}\mathbbm{1}(X_u = A)\,\mathrm{d}l_s^A(X) = 0$. Recall that $V^{c}(t, c(t)) = G(c(t))$ for all $t\in[0,T)$ since $c$ solves \eqref{eq:free-boundary_equation}. Moreover, $V^c(T, z) = G(z)$. Hence, $V^c(t + \rho_{c}, X_{t + \rho_c}) = G(X_{t + \rho_c})$. Therefore, we are able now to derive the following relation from Equations \eqref{eq:V^c_changed} and \eqref{eq:G_changed}, for $(t, x)\in [0, T]\times (-\infty, A)$,
		\begin{align*}
		V^c(t, x) &= \mathbb{E}_{t, x}[e^{-\lambda \rho_c}V^c(t + \rho_c,\ X_{t + \rho_c})] \\
		& \hspace{0.4cm} + \mathbb{E}_{t, x}\left[\int_{t}^{t + \rho_c}e^{-\lambda(u - t)}\left(\left(\alpha\coth(\alpha(T - u)) + \lambda\right)\ X_{u} - \frac{z\alpha}{\sinh(\alpha(T - u))} - \lambda A \right)\mathbb{1}\left(X_{u}\leq c(u)\right)\,\mathrm{d}u\right] \\
		&= \mathbb{E}_{t, x}\left[e^{-\lambda\rho_c}G(X_{t + \rho_c})\right] \\
		&\hspace{0.4cm}  + \mathbb{E}_{t, x}\left[\int_{t}^{t + \rho_c}e^{-\lambda(u - t)}\left(\left(\alpha\coth(\alpha(T - u)) + \lambda\right)\ X_{u} - \frac{z\alpha}{\sinh(\alpha(T - u))} - \lambda A \right)\mathbb{1}\left(X_{u}\leq A\right)\,\mathrm{d}u\right] \\
		& = G(x).
		\end{align*}
		Therefore, we have just proved that $V^{c} = G$ on $C_{2}$.
		
		% V^c <= V !!
		
		Now, define the stopping time
		\begin{align*}
		\tau_{c} := \inf\{0\leq s\leq T-t: X_{t + s} \leq c(t + s)\}
		\end{align*}
		and plug-in it into \eqref{eq:V^c_changed} to obtain the expression
		\begin{align*}
		V^c(t, x) &= e^{-\lambda \tau_c}V^c(t + \tau_c,\ X_{t + \tau_c}) \\
		&\ \ \ + \int_{t}^{t + \tau_c}e^{-\lambda(u - t)}\left(\left(\alpha\coth(\alpha(T - u)) + \lambda\right)\ X_{u} - \frac{z\alpha}{\sinh(\alpha(T - u))} - \lambda A \right)\mathbb{1}\left(X_{u}\leq A\right)\,\mathrm{d}u - M_{\tau_c}^{(1)}.
		\end{align*}
		
		Notice that, due to the definition of $\tau_{c}$, $\mathbbm{1}(X_{t+u} \leq c(t+u)) = 0$ for all $0\leq u < \tau_{c}$ whenever $\tau_c > 0$ (the case $\tau_c = 0$ is trivial). Therefore, the following formula comes after taking $\mathbb{P}_{t, x}$-expectation in the above equation and considering that $V^c = G$ on $C_2$:
		\begin{align*}
		V^{c}(t,x) = \mathbb{E}_{t, x}[e^{-\lambda\tau_c}V^{c}(t+\tau_{c}, X_{t + \tau_{c}})] = \mathbb{E}_{t,x}\left[e^{-\lambda\tau_c}G(X_{t+\tau_{c}})\right],
		\end{align*} 
		for all $(t,x)\in[0,T)\times\mathbb{R}$. Recalling the definition of $V$ from \eqref{eq:OSP}, the above equality leads to
		\begin{align}\label{eq:V^c<V}
		V^{c}(t,x)\leq V(t,x),
		\end{align}
		for all $(t,x)\in[0,T)\times\mathbb{R}$.
		
		% b <= c !!
		
		Take $(t, x)\in C_{2}$ satisfying $x < \min\{b(t), c(t)\}$, and consider the stopping time $\rho_{b}$ defined as
		$$
		\rho_b := \inf\left\{0 \leq s\leq T - t: X_{t + s} \geq b(t + s)\right\}.
		$$
		
		Since $V = G$ on $D$, the following equality holds due to \eqref{eq:pricing_formula} and after noticing that $\mathbbm{1}(X_{t + u} \leq b(t + u)) = 1$ for all $0\leq u < \rho_b$,
		\begin{align*}
		&\mathbb{E}_{t, x}[e^{-\lambda\rho_b}V(t + \rho_b, X_{t + \rho_b})] \\
		&= G(x) -  \mathbb{E}_{t, x}\left[\int_{t}^{t + \rho_b}e^{-\lambda(u - t)}\left(\left(\alpha\coth(\alpha(T - u)) + \lambda\right)\ X_{u} - \frac{z\alpha}{\sinh(\alpha(T - u))} - \lambda A \right)\,\mathrm{d}u\right].
		\end{align*}
		
		On the other hand, we get the next equation after substituting $s$ for $\rho_b$ at \eqref{eq:V^c_changed}  and recalling that $V = G$ on $C_2$,
		\begin{align*}
		&\mathbb{E}_{t, x}[e^{-\lambda\rho_b}V(t+ \rho_b, X_{t + \rho_b})] 
		\\
		&= G(x) - \mathbb{E}_{t, x}\left[\int_{t}^{t + \rho_{c}}e^{-\lambda(u - t)}\left(\left(\alpha\coth(\alpha(T - u)) + \lambda\right)\ X_{u} - \frac{z\alpha}{\sinh(\alpha(T - u))} - \lambda A \right)\mathbb{1}\left(X_{u}\leq c(u)\right)\,\mathrm{d}u\right].
		\end{align*}
		Therefore, we can use \eqref{eq:V^c<V} to merge the two previous equalities into
		\begin{align*}
		\mathbb{E}_{t,x}&\left[\int_{t}^{t + \rho_b}e^{-\lambda(u - t)}\left(\left(\alpha\coth(\alpha(T - u)) + \lambda\right)\ X_{u} - \frac{z\alpha}{\sinh(\alpha(T - u))} - \lambda A \right)\mathbb{1}\left(X_{u}\leq c(u)\right)\,\mathrm{d}u\right]\\
		&\geq \mathbb{E}_{t, x}\left[\int_{t}^{t + \rho_b}e^{-\lambda(u - t)}\left(\left(\alpha\coth(\alpha(T - u)) + \lambda\right)\ X_{u} - \frac{z\alpha}{\sinh(\alpha(T - u))} - \lambda A \right)\,\mathrm{d}u\right],
		\end{align*}
		meaning that $b(t) \leq c(t)$ for all $t\in[0,T]$ since $c$ is continuous. 
		
		% b = c !!!!!  At last!
		
		Suppose there exists a point $t\in(0, T)$ such that $b(t) < c(t)$ and fix $x\in(b(t), c(t))$. Consider the stopping time 
		\begin{align*}
		\tau_{b} := \inf\{0\leq u\leq T-t: X_{t + u} \leq b(t + u)\},
		\end{align*}
		plug-in it into both \eqref{eq:pricing_formula} and \eqref{eq:V^c_changed} replacing $s$, and then  take $\mathbb{P}_{t,x}$-expectation to obtain
		\begin{align*}
		&\mathbb{E}_{t, x}[e^{-\lambda \tau_b}V^c(t + \tau_b, X_{t + \tau_b})] \\ 
		&= \mathbb{E}_{t, x}[e^{-\lambda \tau_b}G(X_{t + \tau_b})] \\
		&= V^c(t, x) - \mathbb{E}_{t, x}\left[\int_{t}^{t + \tau_b}e^{-\lambda(u - t)}\left(\left(\alpha\coth(\alpha(T - u)) + \lambda\right)\ X_{u} - \frac{z\alpha}{\sinh(\alpha(T - u))} - \lambda A \right)\mathbb{1}\left(X_{u}\leq c(u)\right)\,\mathrm{d}u\right]
		\end{align*}
		and
		\begin{align*}
		\mathbb{E}_{t, x}&[e^{-\lambda \tau_b}V(t + \tau_b, X_{t + \tau_b})] = \mathbb{E}_{t, x}[e^{-\lambda \tau_b}G(X_{t + \tau_b})] = V(t, x).
		\end{align*} 
		Thus, from \eqref{eq:V^c<V}, we get
		\begin{align*}
		\mathbb{E}_{t, x}\left[\int_{t}^{t + \tau_b}e^{-\lambda(u - t)}\left(\left(\alpha\coth(\alpha(T - u)) + \lambda\right)\ X_{u} - \frac{z\alpha}{\sinh(\alpha(T - u))} - \lambda A \right)\mathbb{1}\left(X_{u}\leq c(u)\right)\,\mathrm{d}u\right] \leq 0.
		\end{align*}
		
		Using the fact that $x > b(t)$ and the time-continuity of the process $X$, we can state that $\tau_b > 0$. Therefore, the previous inequality can only happen if $\mathbbm{1}(X_{s} \leq c(s)) = 0$ for all $t\leq s\leq t + \tau_b$, meaning that $b(s) \geq c(s)$ for all $t\leq s\leq t + \tau_b$, which contradicts the assumption $b(t) < c(t)$.
	\end{pf}
	\else
	\begin{rmk}
		The integral equation \eqref{eq:free-boundary_equation} has a unique solution among the class of continuous functions of bounded variation lying below $A$. A proof of this statement follows well known arguments based on probability concepts rather than more classical integral equation approaches. This arguments first appeared in \citet{peskir2005ontheamerican} in working with a geometric Brownian motion. We decided not to include a proof here as showing the twists required to address an OUB does not enrich the paper, and its core contribution could be shadowed.
	\end{rmk}
	\fi
	
	\begin{rmk}\label{rmk:theta}
		Denote by and $Y^\theta = \{Y_t^{\theta}\}_{t\geq 0}^{T}$ the OUB process satisfying the stochastic differential equation $\dif Y_t^\theta = \alpha(Y_t^\theta - \theta)\,\dif t + \sigma\,\dif W_t$. Let $Y = \{Y_t\}_{t\geq0}^{T}$ be the OU process on top of that the OUB $X$ in \eqref{eq:OUB_SDE} is build, meaning that $Y$ follows the dynamics $\dif Y_t = \alpha Y_t\,\dif t + \sigma\,\dif W_t$. It is easy to check that $Y_t^{\theta} = Y_t + \theta(1 - e^{-\alpha t})$, from which one can get the relation $X_t^{\theta} = X_t + \theta\left(1 - \frac{\sinh(\alpha (T - t)) + \sinh(\alpha t)}{\sinh(\alpha T)}\right)$, where $X^\theta$ is the bridge of $Y^\theta$ for $Y_T^\theta = Y_T = z$. Recalling \eqref{eq:OUB_rep_int}, it follows that $X_t + \theta\left(1 - \frac{\sinh(\alpha (T - t)) + \sinh(\alpha t)}{\sinh(\alpha T)}\right) = X_t^{\downarrow\theta} + \theta$, where $X_t^{\downarrow\theta}$ is a OUB with $X_0^{\downarrow\theta} = X_0 - \theta$ and $X_T^{\downarrow\theta} = X_T - \theta = z - \theta$. Therefore, if we denote by $V^\theta$ and $b^{\theta}$ the value function and the OSB for the OSP with $x\mapsto (A - x)^+$ and $X^\theta$ as the gain function and the underlying process respectively, one can easily get that $V^\theta(t, x) = V^{\downarrow\theta}(t, x - \theta)$ and  $b^\theta(t) =  b^{\downarrow\theta}(t) + \theta$, where $V^{\downarrow\theta}$ and $b^{\downarrow\theta}$ are the value function and the OSB for the gain $x\mapsto (A - \theta - X^{\downarrow\theta})^+$, given by replacing $A$ and $z$ by $A - \theta$ and $z - \theta$ respectively in \eqref{eq:free-boundary_equation}.
	\end{rmk}
	
	\begin{rmk}
		Denote by $X(\alpha)$ the process in \eqref{eq:OUB_SDE} to emphasize its dependence on $\alpha$. Let $X(0) = \{X_t(0)\}_{t\geq0}^T$ be the $\Pr_{t, x}$-a.s. limit of $X_t(\alpha)$ when $\alpha\rightarrow 0$. According to \eqref{eq:OUB_rep_int}, $X(0)$ is not only well defined, but also distributes like a Brownian bridge\sidenote{1. Could we obtain the same result from the drift convergence in \eqref{eq:OUB_SDE}?\\ 2. I am assuming the limit and the stochastic integral are interchangeable.}. Define $\partial_\alpha X_t(\alpha)$ as the $\Pr_{t, x}$-a.s. limit
		\begin{align*}
		\partial_\alpha X_t(\alpha) := \lim_{h\rightarrow 0}h^{-1}(X_s^{t, x}(\alpha + h) - X_s^{t, x}(\alpha)).
		\end{align*} 
		Directly taking the derivative with respect to $\alpha$ in \eqref{eq:OUB_rep_int}, and noticing that derivating inside the stochastic integral term is allowed since\sidenote{Argument missing!} BLABLABLA, we get the formula
		\begin{align*}
		\partial_\alpha X_{t}(\alpha) &= x\partial_\alpha f(\alpha, T - t, T) + z\partial_\alpha f(\alpha, t, T) \\ &\ \ \ \ + \sigma\int_{0}^{t}\partial_\alpha f(\alpha, T - t, T - u)\,\dif W_u,
		\end{align*}
		where $f(\alpha, s_1, s_2) = \sinh(\alpha s_1)/\sinh(\alpha s_2)$. Name $V_\alpha$ and $V_0$ the value functions associated to $X^{t, x}(\alpha)$ and $X^{t, x}(0)$ respectively, with the gain $G(x) = (A - x)^+$, and let $\tau_\alpha^* = \tau^{*}(t, x)$ be the OST for $V_\alpha(t, x)$ which is sub-optimal for $V_0(t, x)$. Then, use properties \eqref{eq:V_martingale} and \eqref{eq:V_supermartingale} alongside to inequality \eqref{eq:G_inequality} to get
		\begin{align*}
		V_\alpha(t, x) - V_0(t, x) &\leq \E_{t, x}\left[V_\alpha\left(t + \tau_\alpha^*, X_{t + \tau_\alpha^*}(\alpha)\right)\right] - \E\left[V\left(t + \tau_\alpha^*, X_{t + \tau_\alpha^*}(0)\right)\right] \\
		&\leq \E_{t, x}\left[G\left(X_{t + \tau_\alpha^*}(\alpha)\right)\right] - \E\left[G\left(X_{t + \tau_\alpha^*}(0)\right)\right] \\
		&\leq \E_{t, x}\left[\left(X_{t + \tau_\alpha^*}(0) - X_{t + \tau_\alpha^*}(\alpha)\right)^+\right] \\
		&= \E_{t, x}\left[\left(-\alpha\partial_\alpha X_{t + \tau_\alpha^*}(\alpha')\right)^+\right].
		\end{align*} 
		for some $\alpha'\in(\min\{\alpha, 0\}, \max\{\alpha, 0\})$. It is easy to check that $\partial_\alpha f(\alpha, s_1, s_2) \rightarrow 0$ as $\alpha \rightarrow 0$ for all $s_1, s_2$, which alongside the continuity of $\alpha \rightarrow \partial_\alpha f(\alpha, s_1, s_2)$ guarantees that we can use the dominated convergence theorem to pass the limit when $\alpha\rightarrow 0$ inside the expected value to get $\lim_{\alpha\rightarrow 0}V_\alpha \leq V_0$. Following similar arguments but working with $\tau_0^*$ instead of $\tau_\alpha^*$, where $\tau_0^*$ is the OSP for $V_0(t, x)$, we can deduce the reverse inequality $\lim_{\alpha\rightarrow 0}V_\alpha \geq V_0$, meaning that $\lim_{\alpha\rightarrow 0}V_\alpha = V_0$ and consequently $b_ \alpha \rightarrow b_0$ when $\alpha \rightarrow 0$, where $b_ \alpha$ and $b_0$ are the OSB for $V_\alpha$ and $V_0$ respectively.
		
		Therefore we have proof that we can obtain the OSB for a Brownian bridge with final value $z$ and gain function $G(x) = (A - x)^+$ by taking $\alpha \rightarrow 0$ in \eqref{eq:free-boundary_equation}. This is an extension of the work in \citet{d2020discounted}, which finds the OSB for a Brownian bridge within the constrain $A = z$. They point out the complexity of considering $A \neq z$ and shed some light on how to solve it as well as what should we expect on the shape of the OSB. We can easily take $\alpha\rightarrow0$ in \eqref{eq:free-boundary_equation} and notice that the result matches the free boundary equation displayed in \citet{d2020discounted} when $A = z$.  
	\end{rmk}
	
	\begin{thm}
		Let $X = \{X_t\}_{t\geq 0}^T$ be a solution of the SDE
		\begin{align*}
		\dif X_t = \mu(t, X_t)\,\dif t + \sigma\,\dif W_t
		\end{align*}
		with infinitesimal generator $\InfGen$ such that
		$$
		\InfGen f = \partial_t f + \mu\partial_xf + \frac{\sigma^2}{2}\partial_{xx}f. 
		$$
		Let $V(t, x)$ be the value function
		\begin{align}\label{eq:OST_general}
		V(t, x) = \sup_{0\leq \tau \leq T - t}\E\left[e^{-\lambda \tau}G(X_{t + \tau})\right]
		\end{align}
		with gain function $G(x) = (A - x)$. Assume that the following formulas are valid,
		\begin{align}\label{eq:pricing_formula_general}
		V(t, x) = e^{-\lambda (T - t)}\E_{t, x}[G(X_T)] - \int_{t}^{T}\E_{t, x}[(\InfGen G)\mathbf{1}_{(-\infty, b(u))}(X_u)\,\dif u], 
		\end{align}
		and
		\begin{align}\label{eq:free-boundary_equation_general}
		b(t) = A - e^{-\lambda (T - t)}\E_{t, x}[G(X_T)] + \int_{t}^{T}\E_{t, b(t)}[(\InfGen G)\mathbf{1}_{(-\infty, b(u))}(X_u)\,\dif u],
		\end{align}
		where $b$ is the corresponding optimal stopping boundary. Then, equation \eqref{eq:free-boundary_equation_general} has a unique solution among the class of continuous functions, $c:[0, T]\mapsto (-\infty, A]$, of bounded variation.  	
	\end{thm}
	
	\begin{pf}
		
		% Initializations and x -> V^c(t,x) is twice continuously differentiable
		
		Assume we have a function $c:[0,T]\rightarrow \mathbb{R}$ that solves \eqref{eq:free-boundary_equation_general} and define $V^c$ as in \eqref{eq:pricing_formula_general} but with $b$ replaced by $c$. It turns out that \red{the integrand in \eqref{eq:pricing_formula_general} is twice continuously differentiable} with respect to $x$, and therefore we can obtain $\partial_x V^c$ and $\partial_{xx} V^c$, and guarantee their continuity on $[0, T]\times R$, by differentiating inside the integral symbol.
		
		% Infinitesimal of X acting on V^c
		
		%			It is easy to check that $\partial_t V^c(t, x) + \mathbb{L}V^c(t, x) = \lambda V^c(t, x) - \left(\left(\alpha\coth(\alpha(T - u)) + \lambda\right)\ x - \frac{z\alpha}{\sinh(\alpha(T - t))} - \lambda A \right)\mathbb{1}\left(x\leq c(t)\right)$, which, alongside the fact that $V^c$, $\partial_x V^c$ and $\partial_{xx} V^c$ are continuous on $[0, T]\times \R$, we get the continuity of $\partial_t V^c$ on $C_1 \cup C_2$, where
		%			\begin{align*}
		%			C_{1} := \{(t,x) \in [0,T) \times \mathbb{R} : x > c(t)\},\ \ 
		%			C_{2} := \{(t,x) \in [0,T) \times \mathbb{R} : x < c(t)\}.
		%			\end{align*}
		%			
		%			% Itô's Formula for F
		%			
		%			Define the function $F^{(t)}(s, x) := e^{-\lambda s}V^{c}(t + s, x)$ with $s\in[0, T-t)$, $x\in\mathbb{R}$, and consider %the sets
		%			\begin{align*}
		%			C_{1}^t := \{(s,x) \in C_1 : t \leq s < T \},\ \ 
		%			C_{2}^t := \{(s,x) \in C_2 : t \leq s < T \}.
		%			\end{align*}
		%			
		%			Notice that $F^{(t)}$ satisfies the hypothesis of \citet{d2020discounted}[Lemma A2] with $C = C_1^t$ and $D^{\circ} = C_2^t$: $F^{(t)}$, $\partial_xF^{(t)}$, and $\partial_{x^2}F^{(t)}$ are continuous on $[0, T)\R$; it has been proved that $F^{(t)}$ is $\mathcal{C}^{1,2}$ on $C_1^t$ and $C_2^t$; we are assuming that $c$ is a continuous function of bounded variation; and $(\partial_tF^{(t)} + \mathbb{L}_X F^{(t)})(s, x) = -e^{-\lambda s}\left(\left(\alpha\coth(\alpha(T - t - s)) + \lambda\right)\ x - \frac{z\alpha}{\sinh(\alpha(T - t - s))} - \lambda A \right)\mathbb{1}\left(x\leq c(t - s)\right)$ is locally bounded on $C_1^t \cup C_2^t$.
		%			
		%			Thereby, we can use \citet{d2020discounted}[Lemma A2] to obtain the following change-of-variable formula, which is missing the local time term because of the continuity of $F_x$ on $[0, T)\times\mathbb{R}$:
		%			\begin{align}
		%			e^{-\lambda s}&V^c(t + s,\ X_{t + s})\nonumber\\
		%			&= V^c(t, x) - \int_{t}^{t + s}e^{-\lambda(u - t)}\left(\left(\alpha\coth(\alpha(T - u)) + \lambda\right)\ X_{u} - \frac{z\alpha}{\sinh(\alpha(T - u))} - \lambda A \right)\mathbb{1}\left(X_{u}\leq b(u)\right)\,\mathrm{d}u + M_{s}^{(1)}, \label{eq:V^c_changed}
		%			\end{align}
		%			with $M_{s}^{(1)} = \int_{t}^{t + s}e^{-\lambda (u - t)}\sigma \partial_xV^{c}(u, X_{u})\,\mathrm{d}B_{u}$. Notice that $(M_{s}^{(1)})_{s = 0}^{T - t}$ is a martingale under $\mathbb{P}_{t, x}$.
		%			
		%			% Ito's formula for G
		%			
		%			In the same way, we can apply \citet{d2020discounted}[Lemma A2], using now the function $F(s, x) = e^{-\lambda s}G(X_{t + s})$, and taking $C = \{(s, x) \in [0, T - t)\times\mathbb{R} : x > A\}$ and $D^{\circ} = \{(s, x) \in [0, T - t)\times\mathbb{R} :$ $ x < A\}$, thereby getting
		%			\begin{align}\label{eq:G_changed}
		%			e^{-\lambda s}G(X_{t+s}) &= G(x) - \int_{t}^{t + s}e^{-\lambda(u - t)}\left(\left(\alpha\coth(\alpha(T - u)) + \lambda\right)\ X_{u} - \frac{z\alpha}{\sinh(\alpha(T - u))} - \lambda A \right)\mathbb{1}\left(X_{u}\leq A\right)\,\mathrm{d}u \\
		%			&\ \ \ - M_{s}^{(2)} + \frac{1}{2}\int_t^{t + s}e^{-\lambda (u - t)}\mathbbm{1}(X_u = S)\,\mathrm{d}l_s^S(X), \nonumber
		%			\end{align}
		%			where $M_{s}^{(2)} = \sigma\int_{t}^{t + s}e^{-\lambda (u - t)}\mathbbm{1}(X_u < S)\,\mathrm{d}B_{u}$ is a martingale under $\mathbb{P}_{t, x}$, and $l_s^S(X)$ is the local time of $X$ at $A$ up to time $s$.
		%			
		%			% V^c = G on C_2!!
		%			
		%			Consider the following stopping time,
		%			\begin{align}\label{eq:stopping_rho}
		%			\rho_c := \inf\left\{0 \leq s\leq T - t: X_{t + s} \geq c(t + s)\right\}.
		%			\end{align}
		%			Fix $(t, x)$ such that $x \leq c(t)$. As assuming $c(t) < A$ for all $t \in (0, T)$, we can ensure, for  that $\mathbbm{1}(X_{t + s} \leq c(t + s)) = \mathbbm{1}(X_{t + s} \leq A) = 1$ for all $s \in [0, \rho_c)$, as well as $\int_t^{t + s}e^{-\lambda (u - t)}\mathbbm{1}(X_u = A)\,\mathrm{d}l_s^A(X) = 0$. Recall that $V^{c}(t, c(t)) = G(c(t))$ for all $t\in[0,T)$ since $c$ solves \eqref{eq:free-boundary_equation}. Moreover, $V^c(T, z) = G(z)$. Hence, $V^c(t + \rho_{c}, X_{t + \rho_c}) = G(X_{t + \rho_c})$. Therefore, we are able now to derive the following relation from Equations \eqref{eq:V^c_changed} and \eqref{eq:G_changed}, for $(t, x)\in [0, T]\times (-\infty, A)$,
		%			\begin{align*}
		%			V^c(t, x) &= \mathbb{E}_{t, x}[e^{-\lambda \rho_c}V^c(t + \rho_c,\ X_{t + \rho_c})] \\
		%			& \hspace{0.4cm} + \mathbb{E}_{t, x}\left[\int_{t}^{t + \rho_c}e^{-\lambda(u - t)}\left(\left(\alpha\coth(\alpha(T - u)) + \lambda\right)\ X_{u} - \frac{z\alpha}{\sinh(\alpha(T - u))} - \lambda A \right)\mathbb{1}\left(X_{u}\leq c(u)\right)\,\mathrm{d}u\right] \\
		%			&= \mathbb{E}_{t, x}\left[e^{-\lambda\rho_c}G(X_{t + \rho_c})\right] \\
		%			&\hspace{0.4cm}  + \mathbb{E}_{t, x}\left[\int_{t}^{t + \rho_c}e^{-\lambda(u - t)}\left(\left(\alpha\coth(\alpha(T - u)) + \lambda\right)\ X_{u} - \frac{z\alpha}{\sinh(\alpha(T - u))} - \lambda A \right)\mathbb{1}\left(X_{u}\leq A\right)\,\mathrm{d}u\right] \\
		%			& = G(x).
		%			\end{align*}
		%			Therefore, we have just proved that $V^{c} = G$ on $C_{2}$.
		%			
		%			% V^c <= V !!
		%			
		%			Now, define the stopping time
		%			\begin{align*}
		%			\tau_{c} := \inf\{0\leq s\leq T-t: X_{t + s} \leq c(t + s)\}
		%			\end{align*}
		%			and plug-in it into \eqref{eq:V^c_changed} to obtain the expression
		%			\begin{align*}
		%			V^c(t, x) &= e^{-\lambda \tau_c}V^c(t + \tau_c,\ X_{t + \tau_c}) \\
		%			&\ \ \ + \int_{t}^{t + \tau_c}e^{-\lambda(u - t)}\left(\left(\alpha\coth(\alpha(T - u)) + \lambda\right)\ X_{u} - \frac{z\alpha}{\sinh(\alpha(T - u))} - \lambda A \right)\mathbb{1}\left(X_{u}\leq A\right)\,\mathrm{d}u - M_{\tau_c}^{(1)}.
		%			\end{align*}
		%			
		%			Notice that, due to the definition of $\tau_{c}$, $\mathbbm{1}(X_{t+u} \leq c(t+u)) = 0$ for all $0\leq u < \tau_{c}$ whenever $\tau_c > 0$ (the case $\tau_c = 0$ is trivial). Therefore, the following formula comes after taking $\mathbb{P}_{t, x}$-expectation in the above equation and considering that $V^c = G$ on $C_2$:
		%			\begin{align*}
		%			V^{c}(t,x) = \mathbb{E}_{t, x}[e^{-\lambda\tau_c}V^{c}(t+\tau_{c}, X_{t + \tau_{c}})] = \mathbb{E}_{t,x}\left[e^{-\lambda\tau_c}G(X_{t+\tau_{c}})\right],
		%			\end{align*} 
		%			for all $(t,x)\in[0,T)\times\mathbb{R}$. Recalling the definition of $V$ from \eqref{eq:OSP}, the above equality leads to
		%			\begin{align}\label{eq:V^c<V}
		%			V^{c}(t,x)\leq V(t,x),
		%			\end{align}
		%			for all $(t,x)\in[0,T)\times\mathbb{R}$.
		%			
		%			% b <= c !!
		%			
		%			Take $(t, x)\in C_{2}$ satisfying $x < \min\{b(t), c(t)\}$, and consider the stopping time $\rho_{b}$ defined as
		%			$$
		%			\rho_b := \inf\left\{0 \leq s\leq T - t: X_{t + s} \geq b(t + s)\right\}.
		%			$$
		%			
		%			Since $V = G$ on $D$, the following equality holds due to \eqref{eq:pricing_formula} and after noticing that $\mathbbm{1}(X_{t + u} \leq b(t + u)) = 1$ for all $0\leq u < \rho_b$,
		%			\begin{align*}
		%			&\mathbb{E}_{t, x}[e^{-\lambda\rho_b}V(t + \rho_b, X_{t + \rho_b})] \\
		%			&= G(x) -  \mathbb{E}_{t, x}\left[\int_{t}^{t + \rho_b}e^{-\lambda(u - t)}\left(\left(\alpha\coth(\alpha(T - u)) + \lambda\right)\ X_{u} - \frac{z\alpha}{\sinh(\alpha(T - u))} - \lambda A \right)\,\mathrm{d}u\right].
		%			\end{align*}
		%			
		%			On the other hand, we get the next equation after substituting $s$ for $\rho_b$ at \eqref{eq:V^c_changed}  and recalling that $V = G$ on $C_2$,
		%			\begin{align*}
		%			&\mathbb{E}_{t, x}[e^{-\lambda\rho_b}V(t+ \rho_b, X_{t + \rho_b})] 
		%			\\
		%			&= G(x) - \mathbb{E}_{t, x}\left[\int_{t}^{t + \rho_{c}}e^{-\lambda(u - t)}\left(\left(\alpha\coth(\alpha(T - u)) + \lambda\right)\ X_{u} - \frac{z\alpha}{\sinh(\alpha(T - u))} - \lambda A \right)\mathbb{1}\left(X_{u}\leq c(u)\right)\,\mathrm{d}u\right].
		%			\end{align*}
		%			Therefore, we can use \eqref{eq:V^c<V} to merge the two previous equalities into
		%			\begin{align*}
		%			\mathbb{E}_{t,x}&\left[\int_{t}^{t + \rho_b}e^{-\lambda(u - t)}\left(\left(\alpha\coth(\alpha(T - u)) + \lambda\right)\ X_{u} - \frac{z\alpha}{\sinh(\alpha(T - u))} - \lambda A \right)\mathbb{1}\left(X_{u}\leq c(u)\right)\,\mathrm{d}u\right]\\
		%			&\geq \mathbb{E}_{t, x}\left[\int_{t}^{t + \rho_b}e^{-\lambda(u - t)}\left(\left(\alpha\coth(\alpha(T - u)) + \lambda\right)\ X_{u} - \frac{z\alpha}{\sinh(\alpha(T - u))} - \lambda A \right)\,\mathrm{d}u\right],
		%			\end{align*}
		%			meaning that $b(t) \leq c(t)$ for all $t\in[0,T]$ since $c$ is continuous. 
		%			
		%			% b = c !!!!!  At last!
		%			
		%			Suppose there exists a point $t\in(0, T)$ such that $b(t) < c(t)$ and fix $x\in(b(t), c(t))$. Consider the stopping time 
		%			\begin{align*}
		%			\tau_{b} := \inf\{0\leq u\leq T-t: X_{t + u} \leq b(t + u)\},
		%			\end{align*}
		%			plug-in it into both \eqref{eq:pricing_formula} and \eqref{eq:V^c_changed} replacing $s$, and then  take $\mathbb{P}_{t,x}$-expectation to obtain
		%			\begin{align*}
		%			&\mathbb{E}_{t, x}[e^{-\lambda \tau_b}V^c(t + \tau_b, X_{t + \tau_b})] \\ 
		%			&= \mathbb{E}_{t, x}[e^{-\lambda \tau_b}G(X_{t + \tau_b})] \\
		%			&= V^c(t, x) - \mathbb{E}_{t, x}\left[\int_{t}^{t + \tau_b}e^{-\lambda(u - t)}\left(\left(\alpha\coth(\alpha(T - u)) + \lambda\right)\ X_{u} - \frac{z\alpha}{\sinh(\alpha(T - u))} - \lambda A \right)\mathbb{1}\left(X_{u}\leq c(u)\right)\,\mathrm{d}u\right]
		%			\end{align*}
		%			and
		%			\begin{align*}
		%			\mathbb{E}_{t, x}&[e^{-\lambda \tau_b}V(t + \tau_b, X_{t + \tau_b})] = \mathbb{E}_{t, x}[e^{-\lambda \tau_b}G(X_{t + \tau_b})] = V(t, x).
		%			\end{align*} 
		%			Thus, from \eqref{eq:V^c<V}, we get
		%			\begin{align*}
		%			\mathbb{E}_{t, x}\left[\int_{t}^{t + \tau_b}e^{-\lambda(u - t)}\left(\left(\alpha\coth(\alpha(T - u)) + \lambda\right)\ X_{u} - \frac{z\alpha}{\sinh(\alpha(T - u))} - \lambda A \right)\mathbb{1}\left(X_{u}\leq c(u)\right)\,\mathrm{d}u\right] \leq 0.
		%			\end{align*}
		%			
		%			Using the fact that $x > b(t)$ and the time-continuity of the process $X$, we can state that $\tau_b > 0$. Therefore, the previous inequality can only happen if $\mathbbm{1}(X_{s} \leq c(s)) = 0$ for all $t\leq s\leq t + \tau_b$, meaning that $b(s) \geq c(s)$ for all $t\leq s\leq t + \tau_b$, which contradicts the assumption $b(t) < c(t)$.
	\end{pf}
	
	
	\section{Computing the boundary}
	
	\section{Inferring the boundary}
	
	Not knowing the parameters $\alpha$, $\sigma$, and $\theta$ of the OUB is a handicap for a decision maker who can only access to a discrete sample of the process path. In such a case, it is appealing to be able to estimate those parameters in a way that allow us to control the error propagated when computing the boundary based on the estimations. According to \eqref{eq:OUB_rep_int} and Remark \ref{rmk:theta}, the transition density $f(u, \cdot| t, x) = \Pr_{t, x}(X_{u} = \cdot)$ follows a normal distribution with mean $\nu_s^\theta(t, x) = \nu_u(t, x) + \theta\left(1 - \frac{\sinh(\alpha (T - u)) + \sinh(\alpha (u - t))}{\sinh(\alpha( T - t))}\right)$ and variance $\gamma_u^2(t)$, defined at \eqref{eq:mean_Xu} and \eqref{eq:var_Xu} respectively. So the \textit{Markovianity} of the OUB gives us a sequence $\Delta X_n = X_{t_n} - X_{t_{n-1}}$, where $\{t_n\}_{n = 0}^N$ is a partition of $[0, T]$, of independent normally distributed random variables with mean $\nu_{\Delta_n}(t_{n-1}, X_{t_{n - 1}})$ and variance $\gamma_{\Delta_n}^2(t_{n-1})$, with $\nu_u$ and $\gamma_u$ defined at \eqref{eq:mean_Xu} and \eqref{eq:var_Xu} respectively, and $\Delta_n = t_n - t_{n - 1}$, $n = 1,\dots, N$.
	
	\newpage
	
	\bibliography{AmOpOUB}
	\bibliographystyle{apalike}
	
\end{document}